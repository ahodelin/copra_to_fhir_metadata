\chapter{Dokumentation Biosignaldaten-to-\acs{fhir}} \label{ch:docutranf}

\section{Modul Intensivmedizin} \label{sec:docutransfer}

Version 1.0.0

\noindent Neue Profile: \url{https://www.medizininformatik-initiative.de/Kerndatensatz/Modul_Intensivmedizin/IGMIIKDSModulICU.html}


Das Erweiterungsmodul Intensivmedizin oder \ac{icu}, \acs{pdms}/Biosignale spezifiziert akutmedizinische Daten für die Primär- und Sekundärnutzung und hat Bezüge zu den Basismodulen. Ziel der Modellierung dieses Erweiterungsmoduls ist an erster Stelle die Datenabbildung der Intensivmedizin und die Darstellung gleichartiger Daten der Notfallmedizin, stationärer und ambulanter Medizin.

Die benötigte Daten für die Erzeugung der \ac{fhir}-Ressourcen des Erweiterungsmoduls Intensivmedizin befinden sich in der \acsu{copra}-Instanz des \acsu{dw} des \ac{diz} der Universitätsmedizin Mainz. Die Datensätze für die Überführung der Daten liegen in mehreren Tabellen und müssen hierzu im Regelfall zusammengeführt werden.

\begin{itemize}
  \item \texttt{co6\_config\_variables}: Name und Schlüssel der Konfigurationsvariablen oder Biosignalparameter die mit den Namen der Profile zugeordnet wurden.
  \item \texttt{co6\_data\_string}: Pseudonymisierte Patientennummer und Fallnummer, interne Identifikatoren der behandelnden Personen und Fälle.
  \item \texttt{co6\_data\_decimal\_6\_3}: Nummerische Werte der Biosignale, Datum und Uhrzeit der Messung, interner Identifikator der Patienten, Schlüssel der Konfigurationsvariablen
  \item \texttt{co6\_data\_object}: Referenz zu Patient- und Fall-Objekte, Schlüssel der Konfigurationsvariablen.
  \item \texttt{co6\_medic\_pressure}: Systolische, mittlere und diastolische Blutdruckwerte der Biosignalen, Datum und Uhrzeit der Messung, interner Identifikator der Patienten, Schlüssel der Konfigurationsvariablen.
\end{itemize}
\vspace{4mm}

\noindent Version 1.0.0

\noindent Profile: \url{https://www.medizininformatik-initiative.de/Kerndatensatz/Modul_Intensivmedizin/IGMIIKDSModulICU.html}

\noindent Input:
\begin{itemize}
	\item Datensatz aus \texttt{co6\_config\_variables}, \texttt{co6\_data\_decimal\_6\_3} %\\ und \texttt{co6\_data\_string}
\end{itemize}
Output:
\begin{itemize}
	\item \ac{fhir}-Profile der Kategorie \glqq Observation\grqq{} - nummerische Werte %und String Werte
\end{itemize}
\begin{longtable}{|l|l|p{7.5cm}|} 
	\hline
	\rowcolor{lightgray} \multicolumn{3}{|l|}{Data Mapping (inhaltlich) - nummerische Werte} \\ \hline
	id &  & ID in den Tabellen \texttt{co6\_data\_decimal\_6\_3} \\ \hline
	meta & profile & https://medizininformatik-initiative.de/fhir/ext/modul-icu/StructureDefinition/\textbf{Profile\_Name} \\ \hline 
	status &  & final  \\ \hline 
	category & coding & system: http://terminology.hl7.org/ CodeSystem/observation-category \\ 
	\cline{3-3}
	& & code: vital-signs \\ \hline
	code & coding & system: Url von \ac{loinc}, \ac{snomedct}, und / oder \ac{ieee} \\ 
	\cline{3-3} 
	 &  & code: \ac{loinc}, \ac{snomedct}, oder \ac{ieee} \\ \hline
	subject & reference & Pseudonymisierte Patientennummer: \texttt{co6\_medic\_patient.patid} \\ \hline
	valueQuantity & value & Wert der Messung: \texttt{co6\_data\_decimal\_6\_3.val} \\
	\cline{2-3}
	 & system & http://unitsofmeasure.org \\ 
	 \cline{2-3}
	 & code & Mapping auf http://unitsofmeasure.org. (\texttt{mapping\_mii\_co6\_to\_transfer.profile \_unit}) z.B. °C - Cel: \texttt{co6\_config\_variables.unit} \\ \hline
    effectivePeriod & start & Datum und Uhrzeit am Anfang der Messung: \texttt{co6\_data\_decimal\_6\_3.datetimeto} \\ 
    \cline{2-3}
     & end & Datum und Uhrzeit am Ende der Messung: \texttt{co6\_data\_decimal\_6\_3.datetimeto} \\ \hline
\end{longtable}

\vspace{4mm}

\noindent Input:
\begin{itemize}
	\item Datensatz aus \texttt{co6\_ config\_variables}, \texttt{co6\_medic\_patient}, \\ \texttt{co6\_medic\_pressure}
\end{itemize}
Output:
\begin{itemize}
	\item \ac{fhir}-Profile der Kategorie \glqq Observation\grqq{} - Blutdruckmessungen
\end{itemize}
\begin{longtable}{|l|l|p{7cm}|} 
	\hline
	\rowcolor{lightgray} \multicolumn{3}{|l|}{Data Mapping (inhaltlich) - Blutdruckmessungen} \\ \hline
	id &  & ID in der Tabelle \texttt{co6\_medic\_pressure}  \\ \hline
	meta & profile & https://medizininformatik-initiative.de/fhir/ext/modul-icu/StructureDefinition/\textbf{Profile\_Name} \\ \hline 
	status &  & final  \\ \hline 
	category & coding & system: http://terminology.hl7.org/ CodeSystem/observation-category \\ 
	\cline{3-3}
	& & code: vital-signs \\ \hline
	code (Blutdruck) & coding & system: Url von \ac{loinc}, \ac{snomedct}, und / oder \ac{ieee} \\ 
	\cline{3-3} 
	&  & code: \ac{loinc}, \ac{snomedct}, oder \ac{ieee} \\ \hline
	subject & reference & Pseudonymisierte Patientennummer: \texttt{co6\_medic\_patient.patid} \\ \hline
	effectiveDateTime & & Datum und Uhrzeit am Anfang der Messung:  \texttt{co6\_medic\_pressure.datetimeto} \\ \hline
	\multicolumn{3}{|l|}{component} \\ \hline
	code (Systolisch)  & coding & system: Url von \ac{loinc}, \ac{snomedct}, und / oder \ac{ieee} \\ 
	\cline{3-3} 
	&  & code: \ac{loinc}, \ac{snomedct}, oder \ac{ieee} \\ \hline	
	valueQuantity & value & Systolischer Wert: \texttt{co6\_medic\_pressure.systolic} \\
	\cline{2-3}
	& system & http://unitsofmeasure.org \\ 
	\cline{2-3}
	& code & Mapping auf http://unitsofmeasure.org. z.B. mmHg - mm[Hg]: \texttt{co6\_config\_variables.unit} \\ \hline
	code (Mittel)  & coding & system: Url von \ac{loinc}, \ac{snomedct}, und / oder \ac{ieee} \\ 
	\cline{3-3} 
	&  & code: \ac{loinc}, \ac{snomedct}, oder \ac{ieee} \\ \hline	
	valueQuantity & value & Mittlerer Wert: \texttt{co6\_medic\_pressure.mean} \\
	\cline{2-3}
	& system & http://unitsofmeasure.org \\ 
	\cline{2-3}
	& code & Mapping auf http://unitsofmeasure.org. z.B. mmHg - mm[Hg]: \texttt{co6\_config\_variables.unit} \\ \hline
	code (Diastolisch)  & coding & system: Url von \ac{loinc}, \ac{snomedct}, und / oder \ac{ieee} \\ 
	\cline{3-3} 
	&  & code: \ac{loinc}, \ac{snomedct}, oder \ac{ieee} \\ \hline	
	valueQuantity & value & Diastolischer Wert: \texttt{co6\_medic\_pressure.diastolic} \\
	\cline{2-3}
	& system & http://unitsofmeasure.org \\ 
	\cline{2-3}
	& code & Mapping auf http://unitsofmeasure.org. z.B. mmHg - mm[Hg]: \texttt{co6\_config\_variables.unit} \\ \hline
\end{longtable}
\clearpage
