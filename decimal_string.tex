\section{Profile} \label{sec:profil}

\subsection{Arterieller Druck} 
\noindent Version 1.0.0

\noindent Profile: \url{https://www.medizininformatik-initiative.de/Kerndatensatz/Modul_Intensivmedizin/ArteriellerDruckObservation.html}

\noindent Input:
\begin{itemize}
	\item Datensatz aus \texttt{co6\_config\_variables}, \texttt{co6\_medic\_patient}, \\ \texttt{
co6\_data\_decimal\_6\_3}
\end{itemize}
Output:
\begin{itemize}
        \item \ac{fhir}-Ressource der Kategorie \glqq Observation\grqq{}
\end{itemize}
\begin{longtable}{|l|l|p{7.5cm}|}
        \hline
        \rowcolor{lightgray} \multicolumn{3}{|l|}{Data Mapping (inhaltlich)} \\ \hline
        \textbf{id} &  & \texttt{co6\_data\_decimal\_6\_3.id} \\ \hline
	meta & profile & https://medizininformatik-initiative.de/fhir/ext/modul-icu/StructureDefinition/\textbf{
Arterieller-Druck} \\ \hline 
	status &  & in Abstimmung  \\ \hline 
	category & coding & system: \url{http://snomed.info/sct} \\
\cline{3-3}
	& & code: 182744004 \\ \hline
	code & coding & system: Url von \ac{loinc}, \ac{snomedct}, und / oder \ac{ieee} \\ 
	\cline{3-3} 
	 &  & code: \ac{loinc}, \ac{snomedct}, oder \ac{ieee} \\ \hline
	 \textbf{subject}  & reference & Pseudonymisierte Patientennummer: \texttt{co6\_medic\_patient.patid} \\ \hline
	 \textbf{valueQuantity}  & value & Wert der Messung: \texttt{
co6\_data\_decimal\_6\_3.val} \\
        \cline{2-3}
         & system & http://unitsofmeasure.org \\
         \cline{2-3}
         & code &
mm[Hg]
\\ \hline
     \textbf{effectiveDateTime}  & start & Datum und Uhrzeit am Anfang der Messung: \texttt{
co6\_data\_decimal\_6\_3.datetimeto} \\
    \cline{2-3}
     & end & Datum und Uhrzeit am Ende der Messung: \texttt{
co6\_data\_decimal\_6\_3.datetimeto} \\ \hline
\end{longtable}


\subsection{Atemfrequenz} 
\noindent Version 1.0.0

\noindent Profile: \url{https://www.medizininformatik-initiative.de/Kerndatensatz/Modul_Intensivmedizin/AtemfrequenzObservation.html}

\noindent Input:
\begin{itemize}
	\item Datensatz aus \texttt{co6\_config\_variables}, \texttt{co6\_medic\_patient}, \\ \texttt{
co6\_data\_decimal\_6\_3}
\end{itemize}
Output:
\begin{itemize}
        \item \ac{fhir}-Ressource der Kategorie \glqq Observation\grqq{}
\end{itemize}
\begin{longtable}{|l|l|p{7.5cm}|}
        \hline
        \rowcolor{lightgray} \multicolumn{3}{|l|}{Data Mapping (inhaltlich)} \\ \hline
        \textbf{id} &  & \texttt{co6\_data\_decimal\_6\_3.id} \\ \hline
	meta & profile & https://medizininformatik-initiative.de/fhir/ext/modul-icu/StructureDefinition/\textbf{
Atemfrequenz} \\ \hline 
	status &  & in Abstimmung  \\ \hline 
	category & coding & system: \url{http://terminology.hl7.org/CodeSystem/observation-category} \\
\cline{3-3}
	& & code: vital-signs\\ \hline
	code & coding & system: Url von \ac{loinc}, \ac{snomedct}, und / oder \ac{ieee} \\ 
	\cline{3-3} 
	 &  & code: \ac{loinc}, \ac{snomedct}, oder \ac{ieee} \\ \hline
	 \textbf{subject}  & reference & Pseudonymisierte Patientennummer: \texttt{co6\_medic\_patient.patid} \\ \hline
	 \textbf{valueQuantity}  & value & Wert der Messung: \texttt{
co6\_data\_decimal\_6\_3.val} \\
        \cline{2-3}
         & system & http://unitsofmeasure.org \\
         \cline{2-3}
         & code & /min
\\ \hline
     \textbf{effectiveDateTime}  & start & Datum und Uhrzeit am Anfang der Messung: \texttt{
co6\_data\_decimal\_6\_3.datetimeto} \\
    \cline{2-3}
     & end & Datum und Uhrzeit am Ende der Messung: \texttt{
co6\_data\_decimal\_6\_3.datetimeto} \\ \hline
\end{longtable}


\subsection{
Atemzugvolumen-Waehrend-Beatmung} 
\noindent Version 1.0.0

\noindent Profile: \url{https://www.medizininformatik-initiative.de/Kerndatensatz/Modul_Intensivmedizin/Atemzugvolumen-Waehrend-BeatmungObservation.html}

\noindent Input:
\begin{itemize}
	\item Datensatz aus \texttt{co6\_config\_variables}, \texttt{co6\_medic\_patient}, \\ \texttt{
co6\_data\_decimal\_6\_3}
\end{itemize}
Output:
\begin{itemize}
        \item \ac{fhir}-Ressource der Kategorie \glqq Observation\grqq{}
\end{itemize}
\begin{longtable}{|l|l|p{7.5cm}|}
        \hline
        \rowcolor{lightgray} \multicolumn{3}{|l|}{Data Mapping (inhaltlich)} \\ \hline
        \textbf{id} &  & \texttt{co6\_data\_decimal\_6\_3.id} \\ \hline
	meta & profile & https://medizininformatik-initiative.de/fhir/ext/modul-icu/StructureDefinition/\textbf{
Atemzugvolumen-Waehrend-Beatmung} \\ \hline 
	status &  & in Abstimmung  \\ \hline 
	category & coding & system: \url{http://snomed.info/sct} \\
\cline{3-3}
	& & code: 40617009 \\ \hline
	code & coding & system: Url von \ac{loinc}, \ac{snomedct}, und / oder \ac{ieee} \\ 
	\cline{3-3} 
	 &  & code: \ac{loinc}, \ac{snomedct}, oder \ac{ieee} \\ \hline
	 \textbf{subject}  & reference & Pseudonymisierte Patientennummer: \texttt{co6\_medic\_patient.patid} \\ \hline
	 \textbf{valueQuantity}  & value & Wert der Messung: \texttt{
co6\_data\_decimal\_6\_3.val} \\
        \cline{2-3}
         & system & http://unitsofmeasure.org \\
         \cline{2-3}
         & code & mL
\\ \hline
     \textbf{effectiveDateTime}  & start & Datum und Uhrzeit am Anfang der Messung: \texttt{
co6\_data\_decimal\_6\_3.datetimeto} \\
    \cline{2-3}
     & end & Datum und Uhrzeit am Ende der Messung: \texttt{
co6\_data\_decimal\_6\_3.datetimeto} \\ \hline
\end{longtable}


\subsection{Beatmungsvolumen-Pro-Minute-Machineller-Beatmung} 
\noindent Version 1.0.0

\noindent Profile: \url{https://www.medizininformatik-initiative.de/Kerndatensatz/Modul_Intensivmedizin/Beatmungsvolumen-Pro-Minute-Machineller-BeatmungObservation.html}

\noindent Input:
\begin{itemize}
	\item Datensatz aus \texttt{co6\_config\_variables}, \texttt{co6\_medic\_patient}, \\ \texttt{
co6\_data\_decimal\_6\_3}
\end{itemize}
Output:
\begin{itemize}
        \item \ac{fhir}-Ressource der Kategorie \glqq Observation\grqq{}
\end{itemize}
\begin{longtable}{|l|l|p{7.5cm}|}
        \hline
        \rowcolor{lightgray} \multicolumn{3}{|l|}{Data Mapping (inhaltlich)} \\ \hline
        \textbf{id} &  & \texttt{co6\_data\_decimal\_6\_3.id} \\ \hline
	meta & profile & https://medizininformatik-initiative.de/fhir/ext/modul-icu/StructureDefinition/\textbf{
Beatmungsvolumen-Pro-Minute-Machineller-Beatmung} \\ \hline 
	status &  & in Abstimmung  \\ \hline 
	category & coding & system: \url{http://snomed.info/sct} \\
\cline{3-3}
	& & code: 40617009 \\ \hline
	code & coding & system: Url von \ac{loinc}, \ac{snomedct}, und / oder \ac{ieee} \\ 
	\cline{3-3} 
	 &  & code: \ac{loinc}, \ac{snomedct}, oder \ac{ieee} \\ \hline
	 \textbf{subject}  & reference & Pseudonymisierte Patientennummer: \texttt{co6\_medic\_patient.patid} \\ \hline
	 \textbf{valueQuantity}  & value & Wert der Messung: \texttt{
co6\_data\_decimal\_6\_3.val} \\
        \cline{2-3}
         & system & http://unitsofmeasure.org \\
         \cline{2-3}
         & code &
L/min
\\ \hline
     \textbf{effectiveDateTime}  & start & Datum und Uhrzeit am Anfang der Messung: \texttt{
co6\_data\_decimal\_6\_3.datetimeto} \\
    \cline{2-3}
     & end & Datum und Uhrzeit am Ende der Messung: \texttt{
co6\_data\_decimal\_6\_3.datetimeto} \\ \hline
\end{longtable}


\subsection{Blutfluss durch cardiovasculäres Gerät} 
\noindent Version 1.0.0

\noindent Profile: \url{https://www.medizininformatik-initiative.de/Kerndatensatz/Modul_Intensivmedizin/BlutflussdurchcardiovasculresGertObservation.html}

\noindent Input:
\begin{itemize}
	\item Datensatz aus \texttt{co6\_config\_variables}, \texttt{co6\_medic\_patient}, \\ \texttt{
co6\_data\_decimal\_6\_3}
\end{itemize}
Output:
\begin{itemize}
        \item \ac{fhir}-Ressource der Kategorie \glqq Observation\grqq{}
\end{itemize}
\begin{longtable}{|l|l|p{7.5cm}|}
        \hline
        \rowcolor{lightgray} \multicolumn{3}{|l|}{Data Mapping (inhaltlich)} \\ \hline
        \textbf{id} &  & \texttt{co6\_data\_decimal\_6\_3.id} \\ \hline
	meta & profile & https://medizininformatik-initiative.de/fhir/ext/modul-icu/StructureDefinition/\textbf{
Blutfluss-durch-cardiovasculaeres-Geraet} \\ \hline 
	status &  & in Abstimmung  \\ \hline 
	category & coding & system: \url{http://snomed.info/sct} \\
\cline{3-3}
	& & code: 182744004 \\ \hline
	code & coding & system: Url von \ac{loinc}, \ac{snomedct}, und / oder \ac{ieee} \\ 
	\cline{3-3} 
	 &  & code: \ac{loinc}, \ac{snomedct}, oder \ac{ieee} \\ \hline
	 \textbf{subject}  & reference & Pseudonymisierte Patientennummer: \texttt{co6\_medic\_patient.patid} \\ \hline
	 \textbf{valueQuantity}  & value & Wert der Messung: \texttt{
co6\_data\_decimal\_6\_3.val} \\
        \cline{2-3}
         & system & http://unitsofmeasure.org \\
         \cline{2-3}
         & code & L/min
\\ \hline
     \textbf{effectiveDateTime}  & start & Datum und Uhrzeit am Anfang der Messung: \texttt{
co6\_data\_decimal\_6\_3.datetimeto} \\
    \cline{2-3}
     & end & Datum und Uhrzeit am Ende der Messung: \texttt{
co6\_data\_decimal\_6\_3.datetimeto} \\ \hline
\end{longtable}


\subsection{Druckdifferenz Beatmung} 
\noindent Version 1.0.0

\noindent Profile: \url{https://www.medizininformatik-initiative.de/Kerndatensatz/Modul_Intensivmedizin/DruckdifferenzBeatmungObservation.html}

\noindent Input:
\begin{itemize}
	\item Datensatz aus \texttt{co6\_config\_variables}, \texttt{co6\_medic\_patient}, \\ \texttt{
co6\_data\_decimal\_6\_3}
\end{itemize}
Output:
\begin{itemize}
        \item \ac{fhir}-Ressource der Kategorie \glqq Observation\grqq{}
\end{itemize}
\begin{longtable}{|l|l|p{7.5cm}|}
        \hline
        \rowcolor{lightgray} \multicolumn{3}{|l|}{Data Mapping (inhaltlich)} \\ \hline
        \textbf{id} &  & \texttt{co6\_data\_decimal\_6\_3.id} \\ \hline
	meta & profile & https://medizininformatik-initiative.de/fhir/ext/modul-icu/StructureDefinition/\textbf{
Druckdifferenz-Beatmung} \\ \hline 
	status &  & in Abstimmung  \\ \hline 
	category & coding & system: \url{http://snomed.info/sct} \\
\cline{3-3}
	& & code: 40617009 \\ \hline
	code & coding & system: Url von \ac{loinc}, \ac{snomedct}, und / oder \ac{ieee} \\ 
	\cline{3-3} 
	 &  & code: \ac{loinc}, \ac{snomedct}, oder \ac{ieee} \\ \hline
	 \textbf{subject}  & reference & Pseudonymisierte Patientennummer: \texttt{co6\_medic\_patient.patid} \\ \hline
	 \textbf{valueQuantity}  & value & Wert der Messung: \texttt{
co6\_data\_decimal\_6\_3.val} \\
        \cline{2-3}
         & system & http://unitsofmeasure.org \\
         \cline{2-3}
         & code &
cm[H2O]
\\ \hline
     \textbf{effectiveDateTime}  & start & Datum und Uhrzeit am Anfang der Messung: \texttt{
co6\_data\_decimal\_6\_3.datetimeto} \\
    \cline{2-3}
     & end & Datum und Uhrzeit am Ende der Messung: \texttt{
co6\_data\_decimal\_6\_3.datetimeto} \\ \hline
\end{longtable}


\subsection{Einstellung-Einatmungszeit-Beatmung} 
\noindent Version 1.0.0

\noindent Profile: \url{https://www.medizininformatik-initiative.de/Kerndatensatz/Modul_Intensivmedizin/Einstellung-Einatmungszeit-BeatmungObservation.html}

\noindent Input:
\begin{itemize}
	\item Datensatz aus \texttt{co6\_config\_variables}, \texttt{co6\_medic\_patient}, \\ \texttt{
co6\_data\_decimal\_6\_3}
\end{itemize}
Output:
\begin{itemize}
        \item \ac{fhir}-Ressource der Kategorie \glqq Observation\grqq{}
\end{itemize}
\begin{longtable}{|l|l|p{7.5cm}|}
        \hline
        \rowcolor{lightgray} \multicolumn{3}{|l|}{Data Mapping (inhaltlich)} \\ \hline
        \textbf{id} &  & \texttt{co6\_data\_decimal\_6\_3.id} \\ \hline
	meta & profile & https://medizininformatik-initiative.de/fhir/ext/modul-icu/StructureDefinition/\textbf{
Einstellung-Einatmungszeit-Beatmung} \\ \hline 
	status &  & in Abstimmung  \\ \hline 
	category & coding & system: \url{http://snomed.info/sct} \\
\cline{3-3}
	& & code: 40617009\\ \hline
	code & coding & system: Url von \ac{loinc}, \ac{snomedct}, und / oder \ac{ieee} \\ 
	\cline{3-3} 
	 &  & code: \ac{loinc}, \ac{snomedct}, oder \ac{ieee} \\ \hline
	 \textbf{subject}  & reference & Pseudonymisierte Patientennummer: \texttt{co6\_medic\_patient.patid} \\ \hline
	 \textbf{valueQuantity}  & value & Wert der Messung: \texttt{
co6\_data\_decimal\_6\_3.val} \\
        \cline{2-3}
         & system & http://unitsofmeasure.org \\
         \cline{2-3}
         & code &
s
\\ \hline
     \textbf{effectiveDateTime}  & start & Datum und Uhrzeit am Anfang der Messung: \texttt{
co6\_data\_decimal\_6\_3.datetimeto} \\
    \cline{2-3}
     & end & Datum und Uhrzeit am Ende der Messung: \texttt{
co6\_data\_decimal\_6\_3.datetimeto} \\ \hline
\end{longtable}

\subsection{
Exspiratorischer Gasfluss} 
\noindent Version 1.0.0

\noindent Profile: \url{https://www.medizininformatik-initiative.de/Kerndatensatz/Modul_Intensivmedizin/ExspiratorischerGasflussObservation.html}

\noindent Input:
\begin{itemize}
	\item Datensatz aus \texttt{co6\_config\_variables}, \texttt{co6\_medic\_patient}, \\ \texttt{
co6\_data\_decimal\_6\_3}
\end{itemize}
Output:
\begin{itemize}
        \item \ac{fhir}-Ressource der Kategorie \glqq Observation\grqq{}
\end{itemize}
\begin{longtable}{|l|l|p{7.5cm}|}
        \hline
        \rowcolor{lightgray} \multicolumn{3}{|l|}{Data Mapping (inhaltlich)} \\ \hline
        \textbf{id} &  & \texttt{co6\_data\_decimal\_6\_3.id} \\ \hline
	meta & profile & https://medizininformatik-initiative.de/fhir/ext/modul-icu/StructureDefinition/\textbf{
Exspiratorischer-Gasfluss} \\ \hline 
	status &  & in Abstimmung  \\ \hline 
	category & coding & system: \url{http://snomed.info/sct} \\
\cline{3-3}
	& & code: 40617009 \\ \hline
	code & coding & system: Url von \ac{loinc}, \ac{snomedct}, und / oder \ac{ieee} \\ 
	\cline{3-3} 
	 &  & code: \ac{loinc}, \ac{snomedct}, oder \ac{ieee} \\ \hline
	 \textbf{subject}  & reference & Pseudonymisierte Patientennummer: \texttt{co6\_medic\_patient.patid} \\ \hline
	 \textbf{valueQuantity}  & value & Wert der Messung: \texttt{
co6\_data\_decimal\_6\_3.val} \\
        \cline{2-3}
         & system & http://unitsofmeasure.org \\
         \cline{2-3}
         & code & L/min \\ \hline
     \textbf{effectiveDateTime}  & start & Datum und Uhrzeit am Anfang der Messung: \texttt{
co6\_data\_decimal\_6\_3.datetimeto} \\
    \cline{2-3}
     & end & Datum und Uhrzeit am Ende der Messung: \texttt{
co6\_data\_decimal\_6\_3.datetimeto} \\ \hline
\end{longtable}


\subsection{Herzfrequenz} 
\noindent Version 1.0.0

\noindent Profile: \url{https://www.medizininformatik-initiative.de/Kerndatensatz/Modul_Intensivmedizin/HerzfrequenzObservation.html}

\noindent Input:
\begin{itemize}
	\item Datensatz aus \texttt{co6\_config\_variables}, \texttt{co6\_medic\_patient}, \\ \texttt{
co6\_data\_decimal\_6\_3}
\end{itemize}
Output:
\begin{itemize}
        \item \ac{fhir}-Ressource der Kategorie \glqq Observation\grqq{}
\end{itemize}
\begin{longtable}{|l|l|p{7.5cm}|}
        \hline
        \rowcolor{lightgray} \multicolumn{3}{|l|}{Data Mapping (inhaltlich)} \\ \hline
        \textbf{id} &  & \texttt{co6\_data\_decimal\_6\_3.id} \\ \hline
	meta & profile & https://medizininformatik-initiative.de/fhir/ext/modul-icu/StructureDefinition/\textbf{Herzfrequenz} \\ \hline 
	status &  & in Abstimmung  \\ \hline 
	category & coding & system: \url{http://terminology.hl7.org/CodeSystem/observation-category} \\
\cline{3-3}
	& & code: vital-signs\\ \hline
	code & coding & system: Url von \ac{loinc}, \ac{snomedct}, und / oder \ac{ieee} \\ 
	\cline{3-3} 
	 &  & code: \ac{loinc}, \ac{snomedct}, oder \ac{ieee} \\ \hline
	 \textbf{subject}  & reference & Pseudonymisierte Patientennummer: \texttt{co6\_medic\_patient.patid} \\ \hline
	 \textbf{valueQuantity}  & value & Wert der Messung: \texttt{
co6\_data\_decimal\_6\_3.val} \\
        \cline{2-3}
         & system & http://unitsofmeasure.org \\
         \cline{2-3}
         & code & /min \\ \hline
     \textbf{effectiveDateTime}  & start & Datum und Uhrzeit am Anfang der Messung: \texttt{
co6\_data\_decimal\_6\_3.datetimeto} \\
    \cline{2-3}
     & end & Datum und Uhrzeit am Ende der Messung: \texttt{
co6\_data\_decimal\_6\_3.datetimeto} \\ \hline
\end{longtable}


\subsection{Herzzeitvolumen} 
\noindent Version 1.0.0

\noindent Profile: \url{https://www.medizininformatik-initiative.de/Kerndatensatz/Modul_Intensivmedizin/HerzzeitvolumenObservation.html}

\noindent Input:
\begin{itemize}
	\item Datensatz aus \texttt{co6\_config\_variables}, \texttt{co6\_medic\_patient}, \\ \texttt{
co6\_data\_decimal\_6\_3}
\end{itemize}
Output:
\begin{itemize}
        \item \ac{fhir}-Ressource der Kategorie \glqq Observation\grqq{}
\end{itemize}
\begin{longtable}{|l|l|p{7.5cm}|}
        \hline
        \rowcolor{lightgray} \multicolumn{3}{|l|}{Data Mapping (inhaltlich)} \\ \hline
        \textbf{id} &  & \texttt{co6\_data\_decimal\_6\_3.id} \\ \hline
	meta & profile & https://medizininformatik-initiative.de/fhir/ext/modul-icu/StructureDefinition/\textbf{
Herzzeitvolumen} \\ \hline 
	status &  & in Abstimmung  \\ \hline 
	category & coding & system: \url{http://terminology.hl7.org/CodeSystem/observation-category} \\
\cline{3-3}
	& & code: vital-signs \\ \hline
	code & coding & system: Url von \ac{loinc}, \ac{snomedct}, und / oder \ac{ieee} \\ 
	\cline{3-3} 
	 &  & code: \ac{loinc}, \ac{snomedct}, oder \ac{ieee} \\ \hline
	 \textbf{subject}  & reference & Pseudonymisierte Patientennummer: \texttt{co6\_medic\_patient.patid} \\ \hline
	 \textbf{valueQuantity}  & value & Wert der Messung: \texttt{
co6\_data\_decimal\_6\_3.val} \\
        \cline{2-3}
         & system & http://unitsofmeasure.org \\
         \cline{2-3}
         & code &
L/min
\\ \hline
     \textbf{effectiveDateTime}  & start & Datum und Uhrzeit am Anfang der Messung: \texttt{
co6\_data\_decimal\_6\_3.datetimeto} \\
    \cline{2-3}
     & end & Datum und Uhrzeit am Ende der Messung: \texttt{
co6\_data\_decimal\_6\_3.datetimeto} \\ \hline
\end{longtable}


\subsection{Inspiratorischer Gasfluss} 
\noindent Version 1.0.0

\noindent Profile: \url{https://www.medizininformatik-initiative.de/Kerndatensatz/Modul_Intensivmedizin/InspiratorischerGasflussObservation.html}

\noindent Input:
\begin{itemize}
	\item Datensatz aus \texttt{co6\_config\_variables}, \texttt{co6\_medic\_patient}, \\ \texttt{
co6\_data\_decimal\_6\_3}
\end{itemize}
Output:
\begin{itemize}
        \item \ac{fhir}-Ressource der Kategorie \glqq Observation\grqq{}
\end{itemize}
\begin{longtable}{|l|l|p{7.5cm}|}
        \hline
        \rowcolor{lightgray} \multicolumn{3}{|l|}{Data Mapping (inhaltlich)} \\ \hline
        \textbf{id} &  & \texttt{co6\_data\_decimal\_6\_3.id} \\ \hline
	meta & profile & https://medizininformatik-initiative.de/fhir/ext/modul-icu/StructureDefinition/\textbf{
Inspiratorischer-Gasfluss} \\ \hline 
	status &  & in Abstimmung  \\ \hline 
	category & coding & system: \url{http://snomed.info/sct} \\
\cline{3-3}
	& & code: 40617009 \\ \hline
	code & coding & system: Url von \ac{loinc}, \ac{snomedct}, und / oder \ac{ieee} \\ 
	\cline{3-3} 
	 &  & code: \ac{loinc}, \ac{snomedct}, oder \ac{ieee} \\ \hline
	 \textbf{subject}  & reference & Pseudonymisierte Patientennummer: \texttt{co6\_medic\_patient.patid} \\ \hline
	 \textbf{valueQuantity}  & value & Wert der Messung: \texttt{
co6\_data\_decimal\_6\_3.val} \\
        \cline{2-3}
         & system & http://unitsofmeasure.org \\
         \cline{2-3}
         & code & L/min
\\ \hline
     \textbf{effectiveDateTime}  & start & Datum und Uhrzeit am Anfang der Messung: \texttt{
co6\_data\_decimal\_6\_3.datetimeto} \\
    \cline{2-3}
     & end & Datum und Uhrzeit am Ende der Messung: \texttt{
co6\_data\_decimal\_6\_3.datetimeto} \\ \hline
\end{longtable}


\subsection{Intrakranieller Druck (ICP)} 
\noindent Version 1.0.0

\noindent Profile: \url{https://www.medizininformatik-initiative.de/Kerndatensatz/Modul_Intensivmedizin/IntrakraniellerDruckICPObservation.html}

\noindent Input:
\begin{itemize}
	\item Datensatz aus \texttt{co6\_config\_variables}, \texttt{co6\_medic\_patient}, \\ \texttt{
co6\_data\_decimal\_6\_3}
\end{itemize}
Output:
\begin{itemize}
        \item \ac{fhir}-Ressource der Kategorie \glqq Observation\grqq{}
\end{itemize}
\begin{longtable}{|l|l|p{7.5cm}|}
        \hline
        \rowcolor{lightgray} \multicolumn{3}{|l|}{Data Mapping (inhaltlich)} \\ \hline
        \textbf{id} &  & \texttt{co6\_data\_decimal\_6\_3.id} \\ \hline
	meta & profile & https://medizininformatik-initiative.de/fhir/ext/modul-icu/StructureDefinition/\textbf{
Intrakranieller-Druck-(ICP)} \\ \hline 
	status &  & in Abstimmung  \\ \hline 
	category & coding & system: \url{http://terminology.hl7.org/CodeSystem/observation-category} \\
\cline{3-3}
	& & code: vital-signs \\ \hline
	code & coding & system: Url von \ac{loinc}, \ac{snomedct}, und / oder \ac{ieee} \\ 
	\cline{3-3} 
	 &  & code: \ac{loinc}, \ac{snomedct}, oder \ac{ieee} \\ \hline
	 \textbf{subject}  & reference & Pseudonymisierte Patientennummer: \texttt{co6\_medic\_patient.patid} \\ \hline
	 \textbf{valueQuantity}  & value & Wert der Messung: \texttt{
co6\_data\_decimal\_6\_3.val} \\
        \cline{2-3}
         & system & http://unitsofmeasure.org \\
         \cline{2-3}
         & code & mm[Hg] \\ \hline
     \textbf{effectiveDateTime}  & start & Datum und Uhrzeit am Anfang der Messung: \texttt{
co6\_data\_decimal\_6\_3.datetimeto} \\
    \cline{2-3}
     & end & Datum und Uhrzeit am Ende der Messung: \texttt{
co6\_data\_decimal\_6\_3.datetimeto} \\ \hline
\end{longtable}


\subsection{Ionisiertes Kalzium aus Nierenersatzverfahren} 
\noindent Version 1.0.0

\noindent Profile: \url{https://www.medizininformatik-initiative.de/Kerndatensatz/Modul_Intensivmedizin/IonisiertesKalziumausNierenersatzverfahrenObservation.html}

\noindent Input:
\begin{itemize}
	\item Datensatz aus \texttt{co6\_config\_variables}, \texttt{co6\_medic\_patient}, \\ \texttt{
co6\_data\_decimal\_6\_3}
\end{itemize}
Output:
\begin{itemize}
        \item \ac{fhir}-Ressource der Kategorie \glqq Observation\grqq{}
\end{itemize}
\begin{longtable}{|l|l|p{7.5cm}|}
        \hline
        \rowcolor{lightgray} \multicolumn{3}{|l|}{Data Mapping (inhaltlich)} \\ \hline
        \textbf{id} &  & \texttt{co6\_data\_decimal\_6\_3.id} \\ \hline
	meta & profile & https://medizininformatik-initiative.de/fhir/ext/modul-icu/StructureDefinition/\textbf{
Ionisiertes-Kalzium-aus-Nierenersatzverfahren} \\ \hline 
	status &  & in Abstimmung  \\ \hline 
	category & coding & system: \url{http://snomed.info/sct} \\
\cline{3-3}
	& & code: 182744004 \\ \hline
	code & coding & system: Url von \ac{loinc}, \ac{snomedct}, und / oder \ac{ieee} \\ 
	\cline{3-3} 
	 &  & code: \ac{loinc}, \ac{snomedct}, oder \ac{ieee} \\ \hline
	 \textbf{subject}  & reference & Pseudonymisierte Patientennummer: \texttt{co6\_medic\_patient.patid} \\ \hline
	 \textbf{valueQuantity}  & value & Wert der Messung: \texttt{
co6\_data\_decimal\_6\_3.val} \\
        \cline{2-3}
         & system & http://unitsofmeasure.org \\
         \cline{2-3}
         & code & mmol/L \\ \hline
     \textbf{effectiveDateTime}  & start & Datum und Uhrzeit am Anfang der Messung: \texttt{
co6\_data\_decimal\_6\_3.datetimeto} \\
    \cline{2-3}
     & end & Datum und Uhrzeit am Ende der Messung: \texttt{
co6\_data\_decimal\_6\_3.datetimeto} \\ \hline
\end{longtable}


\subsection{
Koerpergewicht} 
\noindent Version 1.0.0

\noindent Profile: \url{https://www.medizininformatik-initiative.de/Kerndatensatz/Modul_Intensivmedizin/KoerpergewichtObservation.html}

\noindent Input:
\begin{itemize}
	\item Datensatz aus \texttt{co6\_config\_variables}, \texttt{co6\_medic\_patient}, \\ \texttt{
co6\_data\_decimal\_6\_3}
\end{itemize}
Output:
\begin{itemize}
        \item \ac{fhir}-Ressource der Kategorie \glqq Observation\grqq{}
\end{itemize}
\begin{longtable}{|l|l|p{7.5cm}|}
        \hline
        \rowcolor{lightgray} \multicolumn{3}{|l|}{Data Mapping (inhaltlich)} \\ \hline
        \textbf{id} &  & \texttt{co6\_data\_decimal\_6\_3.id} \\ \hline
	meta & profile & https://medizininformatik-initiative.de/fhir/ext/modul-icu/StructureDefinition/\textbf{
Koerpergewicht} \\ \hline 
	status &  & in Abstimmung  \\ \hline 
	category & coding & system: \url{http://terminology.hl7.org/CodeSystem/observation-category} \\
\cline{3-3}
	& & code: vital-signs \\ \hline
	code & coding & system: Url von \ac{loinc}, \ac{snomedct}, und / oder \ac{ieee} \\ 
	\cline{3-3} 
	 &  & code: \ac{loinc}, \ac{snomedct}, oder \ac{ieee} \\ \hline
	 \textbf{subject}  & reference & Pseudonymisierte Patientennummer: \texttt{co6\_medic\_patient.patid} \\ \hline
	 \textbf{valueQuantity}  & value & Wert der Messung: \texttt{
co6\_data\_decimal\_6\_3.val} \\
        \cline{2-3}
         & system & http://unitsofmeasure.org \\
         \cline{2-3}
         & code & kg \\ \hline
     \textbf{effectiveDateTime}  & start & Datum und Uhrzeit am Anfang der Messung: \texttt{
co6\_data\_decimal\_6\_3.datetimeto} \\
    \cline{2-3}
     & end & Datum und Uhrzeit am Ende der Messung: \texttt{
co6\_data\_decimal\_6\_3.datetimeto} \\ \hline
\end{longtable}


\subsection{
Koerpergroesse} 
\noindent Version 1.0.0

\noindent Profile: \url{https://www.medizininformatik-initiative.de/Kerndatensatz/Modul_Intensivmedizin/KoerpergroesseObservation.html}

\noindent Input:
\begin{itemize}
	\item Datensatz aus \texttt{co6\_config\_variables}, \texttt{co6\_medic\_patient}, \\ \texttt{
co6\_data\_decimal\_6\_3}
\end{itemize}
Output:
\begin{itemize}
        \item \ac{fhir}-Ressource der Kategorie \glqq Observation\grqq{}
\end{itemize}
\begin{longtable}{|l|l|p{7.5cm}|}
        \hline
        \rowcolor{lightgray} \multicolumn{3}{|l|}{Data Mapping (inhaltlich)} \\ \hline
        \textbf{id} &  & \texttt{co6\_data\_decimal\_6\_3.id} \\ \hline
	meta & profile & https://medizininformatik-initiative.de/fhir/ext/modul-icu/StructureDefinition/\textbf{
Koerpergroesse} \\ \hline 
	status &  & in Abstimmung  \\ \hline 
	category & coding & system: \url{http://terminology.hl7.org/CodeSystem/observation-category} \\
\cline{3-3}
	& & code: vital-signs \\ \hline
	code & coding & system: Url von \ac{loinc}, \ac{snomedct}, und / oder \ac{ieee} \\ 
	\cline{3-3} 
	 &  & code: \ac{loinc}, \ac{snomedct}, oder \ac{ieee} \\ \hline
	 \textbf{subject}  & reference & Pseudonymisierte Patientennummer: \texttt{co6\_medic\_patient.patid} \\ \hline
	 \textbf{valueQuantity}  & value & Wert der Messung: \texttt{
co6\_data\_decimal\_6\_3.val} \\
        \cline{2-3}
         & system & http://unitsofmeasure.org \\
         \cline{2-3}
         & code & cm \\ \hline
     \textbf{effectiveDateTime}  & start & Datum und Uhrzeit am Anfang der Messung: \texttt{
co6\_data\_decimal\_6\_3.datetimeto} \\
    \cline{2-3}
     & end & Datum und Uhrzeit am Ende der Messung: \texttt{
co6\_data\_decimal\_6\_3.datetimeto} \\ \hline
\end{longtable}


\subsection{Koerpertemperatur Kern} 
\noindent Version 1.0.0

\noindent Profile: \url{https://www.medizininformatik-initiative.de/Kerndatensatz/Modul_Intensivmedizin/KoerpertemperaturKernObservation.html}

\noindent Input:
\begin{itemize}
	\item Datensatz aus \texttt{co6\_config\_variables}, \texttt{co6\_medic\_patient}, \\ \texttt{
co6\_data\_decimal\_6\_3}
\end{itemize}
Output:
\begin{itemize}
        \item \ac{fhir}-Ressource der Kategorie \glqq Observation\grqq{}
\end{itemize}
\begin{longtable}{|l|l|p{7.5cm}|}
        \hline
        \rowcolor{lightgray} \multicolumn{3}{|l|}{Data Mapping (inhaltlich)} \\ \hline
        \textbf{id} &  & \texttt{co6\_data\_decimal\_6\_3.id} \\ \hline
	meta & profile & https://medizininformatik-initiative.de/fhir/ext/modul-icu/StructureDefinition/\textbf{
Koerpertemperatur-Kern} \\ \hline 
	status &  & in Abstimmung  \\ \hline 
	category & coding & system: \url{http://terminology.hl7.org/CodeSystem/observation-category} \\
\cline{3-3}
	& & code: vital-signs \\ \hline
	code & coding & system: Url von \ac{loinc}, \ac{snomedct}, und / oder \ac{ieee} \\ 
	\cline{3-3} 
	 &  & code: \ac{loinc}, \ac{snomedct}, oder \ac{ieee} \\ \hline
	 \textbf{subject}  & reference & Pseudonymisierte Patientennummer: \texttt{co6\_medic\_patient.patid} \\ \hline
	 \textbf{valueQuantity}  & value & Wert der Messung: \texttt{
co6\_data\_decimal\_6\_3.val} \\
        \cline{2-3}
         & system & http://unitsofmeasure.org \\
         \cline{2-3}
         & code & Cel \\ \hline
     \textbf{effectiveDateTime}  & start & Datum und Uhrzeit am Anfang der Messung: \texttt{
co6\_data\_decimal\_6\_3.datetimeto} \\
    \cline{2-3}
     & end & Datum und Uhrzeit am Ende der Messung: \texttt{
co6\_data\_decimal\_6\_3.datetimeto} \\ \hline
\end{longtable}


\subsection{Kopfumfang} 
\noindent Version 1.0.0

\noindent Profile: \url{https://www.medizininformatik-initiative.de/Kerndatensatz/Modul_Intensivmedizin/KopfumfangObservation.html}

\noindent Input:
\begin{itemize}
	\item Datensatz aus \texttt{co6\_config\_variables}, \texttt{co6\_medic\_patient}, \\ \texttt{
co6\_data\_decimal\_6\_3}
\end{itemize}
Output:
\begin{itemize}
        \item \ac{fhir}-Ressource der Kategorie \glqq Observation\grqq{}
\end{itemize}
\begin{longtable}{|l|l|p{7.5cm}|}
        \hline
        \rowcolor{lightgray} \multicolumn{3}{|l|}{Data Mapping (inhaltlich)} \\ \hline
        \textbf{id} &  & \texttt{co6\_data\_decimal\_6\_3.id} \\ \hline
	meta & profile & https://medizininformatik-initiative.de/fhir/ext/modul-icu/StructureDefinition/\textbf{
Kopfumfang} \\ \hline 
	status &  & in Abstimmung  \\ \hline 
	category & coding & system: \url{http://terminology.hl7.org/CodeSystem/observation-category} \\
\cline{3-3}
	& & code: vital-signs \\ \hline
	code & coding & system: Url von \ac{loinc}, \ac{snomedct}, und / oder \ac{ieee} \\ 
	\cline{3-3} 
	 &  & code: \ac{loinc}, \ac{snomedct}, oder \ac{ieee} \\ \hline
	 \textbf{subject}  & reference & Pseudonymisierte Patientennummer: \texttt{co6\_medic\_patient.patid} \\ \hline
	 \textbf{valueQuantity}  & value & Wert der Messung: \texttt{
co6\_data\_decimal\_6\_3.val} \\
        \cline{2-3}
         & system & http://unitsofmeasure.org \\
         \cline{2-3}
         & code & cm \\ \hline
     \textbf{effectiveDateTime}  & start & Datum und Uhrzeit am Anfang der Messung: \texttt{
co6\_data\_decimal\_6\_3.datetimeto} \\
    \cline{2-3}
     & end & Datum und Uhrzeit am Ende der Messung: \texttt{
co6\_data\_decimal\_6\_3.datetimeto} \\ \hline
\end{longtable}


\subsection{
Linksatrialer Druck} 
\noindent Version 1.0.0

\noindent Profile: \url{https://www.medizininformatik-initiative.de/Kerndatensatz/Modul_Intensivmedizin/LinksatrialerDruckObservation.html}

\noindent Input:
\begin{itemize}
	\item Datensatz aus \texttt{co6\_config\_variables}, \texttt{co6\_medic\_patient}, \\ \texttt{
co6\_data\_decimal\_6\_3}
\end{itemize}
Output:
\begin{itemize}
        \item \ac{fhir}-Ressource der Kategorie \glqq Observation\grqq{}
\end{itemize}
\begin{longtable}{|l|l|p{7.5cm}|}
        \hline
        \rowcolor{lightgray} \multicolumn{3}{|l|}{Data Mapping (inhaltlich)} \\ \hline
        \textbf{id} &  & \texttt{co6\_data\_decimal\_6\_3.id} \\ \hline
	meta & profile & https://medizininformatik-initiative.de/fhir/ext/modul-icu/StructureDefinition/\textbf{
Linksatrialer-Druck} \\ \hline 
	status &  & in Abstimmung  \\ \hline 
	category & coding & system: \url{http://terminology.hl7.org/CodeSystem/observation-category} \\
\cline{3-3}
	& & code: vital-signs \\ \hline
	code & coding & system: Url von \ac{loinc}, \ac{snomedct}, und / oder \ac{ieee} \\ 
	\cline{3-3} 
	 &  & code: \ac{loinc}, \ac{snomedct}, oder \ac{ieee} \\ \hline
	 \textbf{subject}  & reference & Pseudonymisierte Patientennummer: \texttt{co6\_medic\_patient.patid} \\ \hline
	 \textbf{valueQuantity}  & value & Wert der Messung: \texttt{
co6\_data\_decimal\_6\_3.val} \\
        \cline{2-3}
         & system & http://unitsofmeasure.org \\
         \cline{2-3}
         & code & mm[Hg] \\ \hline
     \textbf{effectiveDateTime}  & start & Datum und Uhrzeit am Anfang der Messung: \texttt{
co6\_data\_decimal\_6\_3.datetimeto} \\
    \cline{2-3}
     & end & Datum und Uhrzeit am Ende der Messung: \texttt{
co6\_data\_decimal\_6\_3.datetimeto} \\ \hline
\end{longtable}


\subsection{Linksventrikulärer Schlagvolumenindex} 
\noindent Version 1.0.0

\noindent Profile: \url{https://www.medizininformatik-initiative.de/Kerndatensatz/Modul_Intensivmedizin/LinksventrikulrerSchlagvolumenindexObservation2.html}

\noindent Input:
\begin{itemize}
	\item Datensatz aus \texttt{co6\_config\_variables}, \texttt{co6\_medic\_patient}, \\ \texttt{
co6\_data\_decimal\_6\_3}
\end{itemize}
Output:
\begin{itemize}
        \item \ac{fhir}-Ressource der Kategorie \glqq Observation\grqq{}
\end{itemize}
\begin{longtable}{|l|l|p{7.5cm}|}
        \hline
        \rowcolor{lightgray} \multicolumn{3}{|l|}{Data Mapping (inhaltlich)} \\ \hline
        \textbf{id} &  & \texttt{co6\_data\_decimal\_6\_3.id} \\ \hline
	meta & profile & https://medizininformatik-initiative.de/fhir/ext/modul-icu/StructureDefinition/\textbf{
Linksventrikulaerer-Schlagvolumenindex} \\ \hline 
	status &  & in Abstimmung  \\ \hline 
	category & coding & system: \url{http://terminology.hl7.org/CodeSystem/observation-category} \\
\cline{3-3}
	& & code: vital-signs \\ \hline
	code & coding & system: Url von \ac{loinc}, \ac{snomedct}, und / oder \ac{ieee} \\ 
	\cline{3-3} 
	 &  & code: \ac{loinc}, \ac{snomedct}, oder \ac{ieee} \\ \hline
	 \textbf{subject}  & reference & Pseudonymisierte Patientennummer: \texttt{co6\_medic\_patient.patid} \\ \hline
	 \textbf{valueQuantity}  & value & Wert der Messung: \texttt{
co6\_data\_decimal\_6\_3.val} \\
        \cline{2-3}
         & system & http://unitsofmeasure.org \\
         \cline{2-3}
         & code & mL/m2 \\ \hline
     \textbf{effectiveDateTime}  & start & Datum und Uhrzeit am Anfang der Messung: \texttt{
co6\_data\_decimal\_6\_3.datetimeto} \\
    \cline{2-3}
     & end & Datum und Uhrzeit am Ende der Messung: \texttt{co6\_data\_decimal\_6\_3.datetimeto} \\ \hline
\end{longtable}


\subsection{Mechanische Atemfrequenz Beatmet} 
\noindent Version 1.0.0

\noindent Profile: \url{https://www.medizininformatik-initiative.de/Kerndatensatz/Modul_Intensivmedizin/Mechanische-Atemfrequenz-BeatmetObservation.html}

\noindent Input:

\begin{itemize}
	\item Datensatz aus \texttt{co6\_config\_variables}, \texttt{co6\_medic\_patient}, \\ \texttt{
co6\_data\_decimal\_6\_3}
\end{itemize}
Output:
\begin{itemize}
        \item \ac{fhir}-Ressource der Kategorie \glqq Observation\grqq{}
\end{itemize}
\begin{longtable}{|l|l|p{7.5cm}|}
        \hline
        \rowcolor{lightgray} \multicolumn{3}{|l|}{Data Mapping (inhaltlich)} \\ \hline
        \textbf{id} &  & \texttt{co6\_data\_decimal\_6\_3.id} \\ \hline
	meta & profile & https://medizininformatik-initiative.de/fhir/ext/modul-icu/StructureDefinition/\textbf{
Mechanische-Atemfrequenz-Beatmet} \\ \hline 
	status &  & in Abstimmung  \\ \hline 
	category & coding & system: \url{http://snomed.info/sct} \\
\cline{3-3}
	& & code: 40617009 \\ \hline
	code & coding & system: Url von \ac{loinc}, \ac{snomedct}, und / oder \ac{ieee} \\ 
	\cline{3-3} 
	 &  & code: \ac{loinc}, \ac{snomedct}, oder \ac{ieee} \\ \hline
	 \textbf{subject}  & reference & Pseudonymisierte Patientennummer: \texttt{co6\_medic\_patient.patid} \\ \hline
	 \textbf{valueQuantity}  & value & Wert der Messung: \texttt{
co6\_data\_decimal\_6\_3.val} \\
        \cline{2-3}
         & system & http://unitsofmeasure.org \\
         \cline{2-3}
         & code & {Breaths}/min \\ \hline
     \textbf{effectiveDateTime}  & start & Datum und Uhrzeit am Anfang der Messung: \texttt{
co6\_data\_decimal\_6\_3.datetimeto} \\
    \cline{2-3}
     & end & Datum und Uhrzeit am Ende der Messung: \texttt{
co6\_data\_decimal\_6\_3.datetimeto} \\ \hline
\end{longtable}


\subsection{Mittlerer Beatmungsdruck} 
\noindent Version 1.0.0

\noindent Profile: \url{https://www.medizininformatik-initiative.de/Kerndatensatz/Modul_Intensivmedizin/MittlererBeatmungsdruckObservation.html}

\noindent Input:
\begin{itemize}
	\item Datensatz aus \texttt{co6\_config\_variables}, \texttt{co6\_medic\_patient}, \\ \texttt{
co6\_data\_decimal\_6\_3}
\end{itemize}
Output:
\begin{itemize}
        \item \ac{fhir}-Ressource der Kategorie \glqq Observation\grqq{}
\end{itemize}
\begin{longtable}{|l|l|p{7.5cm}|}
        \hline
        \rowcolor{lightgray} \multicolumn{3}{|l|}{Data Mapping (inhaltlich)} \\ \hline
        \textbf{id} &  & \texttt{co6\_data\_decimal\_6\_3.id} \\ \hline
	meta & profile & https://medizininformatik-initiative.de/fhir/ext/modul-icu/StructureDefinition/\textbf{
Mittlerer-Beatmungsdruck} \\ \hline 
	status &  & in Abstimmung  \\ \hline 
	category & coding & system: \url{http://snomed.info/sct} \\
\cline{3-3}
	& & code: 40617009 \\ \hline
	code & coding & system: Url von \ac{loinc}, \ac{snomedct}, und / oder \ac{ieee} \\ 
	\cline{3-3} 
	 &  & code: \ac{loinc}, \ac{snomedct}, oder \ac{ieee} \\ \hline
	 \textbf{subject}  & reference & Pseudonymisierte Patientennummer: \texttt{co6\_medic\_patient.patid} \\ \hline
	 \textbf{valueQuantity}  & value & Wert der Messung: \texttt{co6\_data\_decimal\_6\_3.val} \\
        \cline{2-3}
         & system & http://unitsofmeasure.org \\
         \cline{2-3}
         & code & cm[H2O] \\ \hline
     \textbf{effectiveDateTime}  & start & Datum und Uhrzeit am Anfang der Messung: \texttt{
co6\_data\_decimal\_6\_3.datetimeto} \\
    \cline{2-3}
     & end & Datum und Uhrzeit am Ende der Messung: \texttt{
co6\_data\_decimal\_6\_3.datetimeto} \\ \hline
\end{longtable}


\subsection{Positiv-endexpiratorischer Druck} 
\noindent Version 1.0.0

\noindent Profile: \url{https://www.medizininformatik-initiative.de/Kerndatensatz/Modul_Intensivmedizin/Positiv-endexpiratorischerDruckObservation.html}

\noindent Input:
\begin{itemize}
	\item Datensatz aus \texttt{co6\_config\_variables}, \texttt{co6\_medic\_patient}, \\ \texttt{
co6\_data\_decimal\_6\_3}
\end{itemize}
Output:
\begin{itemize}
        \item \ac{fhir}-Ressource der Kategorie \glqq Observation\grqq{}
\end{itemize}
\begin{longtable}{|l|l|p{7.5cm}|}
        \hline
        \rowcolor{lightgray} \multicolumn{3}{|l|}{Data Mapping (inhaltlich)} \\ \hline
        \textbf{id} &  & \texttt{co6\_data\_decimal\_6\_3.id} \\ \hline
	meta & profile & https://medizininformatik-initiative.de/fhir/ext/modul-icu/StructureDefinition/\textbf{
Positiv-endexpiratorischer-Druck} \\ \hline 
	status &  & in Abstimmung  \\ \hline 
	category & coding & system: \url{http://snomed.info/sct} \\
\cline{3-3}
	& & code: 40617009 \\ \hline
	code & coding & system: Url von \ac{loinc}, \ac{snomedct}, und / oder \ac{ieee} \\ 
	\cline{3-3} 
	 &  & code: \ac{loinc}, \ac{snomedct}, oder \ac{ieee} \\ \hline
	 \textbf{subject}  & reference & Pseudonymisierte Patientennummer: \texttt{co6\_medic\_patient.patid} \\ \hline
	 \textbf{valueQuantity}  & value & Wert der Messung: \texttt{co6\_data\_decimal\_6\_3.val} \\
        \cline{2-3}
         & system & http://unitsofmeasure.org \\
         \cline{2-3}
         & code & cm[H2O] \\ \hline
     \textbf{effectiveDateTime}  & start & Datum und Uhrzeit am Anfang der Messung: \texttt{
co6\_data\_decimal\_6\_3.datetimeto} \\
    \cline{2-3}
     & end & Datum und Uhrzeit am Ende der Messung: \texttt{
co6\_data\_decimal\_6\_3.datetimeto} \\ \hline
\end{longtable}


\subsection{
Pulmonalarterieller wedge Blutdruck} 
\noindent Version 1.0.0

\noindent Profile: \url{https://www.medizininformatik-initiative.de/Kerndatensatz/Modul_Intensivmedizin/PulmonalarteriellerwedgeBlutdruckObservation.html}

\noindent Input:
\begin{itemize}
	\item Datensatz aus \texttt{co6\_config\_variables}, \texttt{co6\_medic\_patient}, \\ \texttt{
co6\_data\_decimal\_6\_3}
\end{itemize}
Output:
\begin{itemize}
        \item \ac{fhir}-Ressource der Kategorie \glqq Observation\grqq{}
\end{itemize}
\begin{longtable}{|l|l|p{7.5cm}|}
        \hline
        \rowcolor{lightgray} \multicolumn{3}{|l|}{Data Mapping (inhaltlich)} \\ \hline
        \textbf{id} &  & \texttt{co6\_data\_decimal\_6\_3.id} \\ \hline
	meta & profile & https://medizininformatik-initiative.de/fhir/ext/modul-icu/StructureDefinition/\textbf{
Pulmonalarterieller-wedge-Blutdruck} \\ \hline 
	status &  & in Abstimmung  \\ \hline 
	category & coding & system: \url{http://terminology.hl7.org/CodeSystem/observation-category} \\
\cline{3-3}
	& & code: vital-signs\\ \hline
	code & coding & system: Url von \ac{loinc}, \ac{snomedct}, und / oder \ac{ieee} \\ 
	\cline{3-3} 
	 &  & code: \ac{loinc}, \ac{snomedct}, oder \ac{ieee} \\ \hline
	 \textbf{subject}  & reference & Pseudonymisierte Patientennummer: \texttt{co6\_medic\_patient.patid} \\ \hline
	 \textbf{valueQuantity}  & value & Wert der Messung: \texttt{
co6\_data\_decimal\_6\_3.val} \\
        \cline{2-3}
         & system & http://unitsofmeasure.org \\
         \cline{2-3}
         & code & mm[Hg] \\ \hline
     \textbf{effectiveDateTime}  & start & Datum und Uhrzeit am Anfang der Messung: \texttt{
co6\_data\_decimal\_6\_3.datetimeto} \\
    \cline{2-3}
     & end & Datum und Uhrzeit am Ende der Messung: \texttt{
co6\_data\_decimal\_6\_3.datetimeto} \\ \hline
\end{longtable}


\subsection{Sauerstofffraktion} 
\noindent Version 1.0.0

\noindent Profile: \url{https://www.medizininformatik-initiative.de/Kerndatensatz/Modul_Intensivmedizin/SauerstofffraktionObservation.html}

\noindent Input:
\begin{itemize}
	\item Datensatz aus \texttt{co6\_config\_variables}, \texttt{co6\_medic\_patient}, \\ \texttt{
co6\_data\_decimal\_6\_3}
\end{itemize}
Output:
\begin{itemize}
        \item \ac{fhir}-Ressource der Kategorie \glqq Observation\grqq{}
\end{itemize}
\begin{longtable}{|l|l|p{7.5cm}|}
        \hline
        \rowcolor{lightgray} \multicolumn{3}{|l|}{Data Mapping (inhaltlich)} \\ \hline
        \textbf{id} &  & \texttt{co6\_data\_decimal\_6\_3.id} \\ \hline
	meta & profile & https://medizininformatik-initiative.de/fhir/ext/modul-icu/StructureDefinition/\textbf{
Sauerstofffraktion} \\ \hline 
	status &  & in Abstimmung  \\ \hline 
	category & coding & system: \url{http://snomed.info/sct} \\
\cline{3-3}
	& & code: 40617009 \\ \hline
	code & coding & system: Url von \ac{loinc}, \ac{snomedct}, und / oder \ac{ieee} \\ 
	\cline{3-3} 
	 &  & code: \ac{loinc}, \ac{snomedct}, oder \ac{ieee} \\ \hline
	 \textbf{subject}  & reference & Pseudonymisierte Patientennummer: \texttt{co6\_medic\_patient.patid} \\ \hline
	 \textbf{valueQuantity}  & value & Wert der Messung: \texttt{
co6\_data\_decimal\_6\_3.val} \\
        \cline{2-3}
         & system & http://unitsofmeasure.org \\
         \cline{2-3}
         & code & \\ \hline
     \textbf{effectiveDateTime}  & start & Datum und Uhrzeit am Anfang der Messung: \texttt{co6\_data\_decimal\_6\_3.datetimeto} \\
    \cline{2-3}
     & end & Datum und Uhrzeit am Ende der Messung: \texttt{co6\_data\_decimal\_6\_3.datetimeto} \\ \hline
\end{longtable}


\subsection{Sauerstofffraktion eingestellt} 
\noindent Version 1.0.0

\noindent Profile: \url{https://www.medizininformatik-initiative.de/Kerndatensatz/Modul_Intensivmedizin/SauerstofffraktioneingestelltObservation.html}

\noindent Input:
\begin{itemize}
	\item Datensatz aus \texttt{co6\_config\_variables}, \texttt{co6\_medic\_patient}, \\ \texttt{
co6\_data\_decimal\_6\_3}
\end{itemize}
Output:
\begin{itemize}
        \item \ac{fhir}-Ressource der Kategorie \glqq Observation\grqq{}
\end{itemize}
\begin{longtable}{|l|l|p{7.5cm}|}
        \hline
        \rowcolor{lightgray} \multicolumn{3}{|l|}{Data Mapping (inhaltlich)} \\ \hline
        \textbf{id} &  & \texttt{co6\_data\_decimal\_6\_3.id} \\ \hline
	meta & profile & https://medizininformatik-initiative.de/fhir/ext/modul-icu/StructureDefinition/\textbf{
Sauerstofffraktion-eingestellt} \\ \hline 
	status &  & in Abstimmung  \\ \hline 
	category & coding & system: \url{http://snomed.info/sct} \\
\cline{3-3}
	& & code: 40617009 \\ \hline
	code & coding & system: Url von \ac{loinc}, \ac{snomedct}, und / oder \ac{ieee} \\ 
	\cline{3-3} 
	 &  & code: \ac{loinc}, \ac{snomedct}, oder \ac{ieee} \\ \hline
	 \textbf{subject}  & reference & Pseudonymisierte Patientennummer: \texttt{co6\_medic\_patient.patid} \\ \hline
	 \textbf{valueQuantity}  & value & Wert der Messung: \texttt{
co6\_data\_decimal\_6\_3.val} \\
        \cline{2-3}
         & system & http://unitsofmeasure.org \\
         \cline{2-3}
         & code & \\ \hline
     \textbf{effectiveDateTime}  & start & Datum und Uhrzeit am Anfang der Messung: \texttt{
co6\_data\_decimal\_6\_3.datetimeto} \\
    \cline{2-3}
     & end & Datum und Uhrzeit am Ende der Messung: \texttt{
co6\_data\_decimal\_6\_3.datetimeto} \\ \hline
\end{longtable}


\subsection{Sauerstoffgasfluss} 
\noindent Version 1.0.0

\noindent Profile: \url{https://www.medizininformatik-initiative.de/Kerndatensatz/Modul_Intensivmedizin/SauerstoffgasflussObservation.html}

\noindent Input:
\begin{itemize}
	\item Datensatz aus \texttt{co6\_config\_variables}, \texttt{co6\_medic\_patient}, \\ \texttt{
co6\_data\_decimal\_6\_3}
\end{itemize}
Output:
\begin{itemize}
        \item \ac{fhir}-Ressource der Kategorie \glqq Observation\grqq{}
\end{itemize}
\begin{longtable}{|l|l|p{7.5cm}|}
        \hline
        \rowcolor{lightgray} \multicolumn{3}{|l|}{Data Mapping (inhaltlich)} \\ \hline
        \textbf{id} &  & \texttt{co6\_data\_decimal\_6\_3.id} \\ \hline
	meta & profile & https://medizininformatik-initiative.de/fhir/ext/modul-icu/StructureDefinition/\textbf{
Sauerstoffgasfluss} \\ \hline 
	status &  & in Abstimmung  \\ \hline 
	category & coding & system: \url{http://snomed.info/sct} \\
\cline{3-3}
	& & code: 182744004 \\ \hline
	code & coding & system: Url von \ac{loinc}, \ac{snomedct}, und / oder \ac{ieee} \\ 
	\cline{3-3} 
	 &  & code: \ac{loinc}, \ac{snomedct}, oder \ac{ieee} \\ \hline
	 \textbf{subject}  & reference & Pseudonymisierte Patientennummer: \texttt{co6\_medic\_patient.patid} \\ \hline
	 \textbf{valueQuantity}  & value & Wert der Messung: \texttt{
co6\_data\_decimal\_6\_3.val} \\
        \cline{2-3}
         & system & http://unitsofmeasure.org \\
         \cline{2-3}
         & code & L/min \\ \hline
     \textbf{effectiveDateTime}  & start & Datum und Uhrzeit am Anfang der Messung: \texttt{
co6\_data\_decimal\_6\_3.datetimeto} \\
    \cline{2-3}
     & end & Datum und Uhrzeit am Ende der Messung: \texttt{
co6\_data\_decimal\_6\_3.datetimeto} \\ \hline
\end{longtable}


\subsection{Sauerstoffsättigung im art. Blut durch Pulsoxymetrie} 
\noindent Version 1.0.0

\noindent Profile: \url{https://www.medizininformatik-initiative.de/Kerndatensatz/Modul_Intensivmedizin/Sauerstoffsttigungimart.BlutdurchPulsoxymetrieObs.html}

\noindent Input:
\begin{itemize}
	\item Datensatz aus \texttt{co6\_config\_variables}, \texttt{co6\_medic\_patient}, \\ \texttt{
co6\_data\_decimal\_6\_3}
\end{itemize}
Output:
\begin{itemize}
        \item \ac{fhir}-Ressource der Kategorie \glqq Observation\grqq{}
\end{itemize}
\begin{longtable}{|l|l|p{7.5cm}|}
        \hline
        \rowcolor{lightgray} \multicolumn{3}{|l|}{Data Mapping (inhaltlich)} \\ \hline
        \textbf{id} &  & \texttt{co6\_data\_decimal\_6\_3.id} \\ \hline
	meta & profile & https://medizininformatik-initiative.de/fhir/ext/modul-icu/StructureDefinition/\textbf{
Sauerstoffsaettigung-im-art.-Blut-durch-Pulsoxymetrie} \\ \hline 
	status &  & in Abstimmung  \\ \hline 
	category & coding & system: \url{http://terminology.hl7.org/CodeSystem/observation-category} \\
\cline{3-3}
	& & code: vital-signs \\ \hline
	code & coding & system: Url von \ac{loinc}, \ac{snomedct}, und / oder \ac{ieee} \\ 
	\cline{3-3} 
	 &  & code: \ac{loinc}, \ac{snomedct}, oder \ac{ieee} \\ \hline
	 \textbf{subject}  & reference & Pseudonymisierte Patientennummer: \texttt{co6\_medic\_patient.patid} \\ \hline
	 \textbf{valueQuantity}  & value & Wert der Messung: \texttt{
co6\_data\_decimal\_6\_3.val} \\
        \cline{2-3}
         & system & http://unitsofmeasure.org \\
         \cline{2-3}
         & code & \% \\ \hline
     \textbf{effectiveDateTime}  & start & Datum und Uhrzeit am Anfang der Messung: \texttt{
co6\_data\_decimal\_6\_3.datetimeto} \\
    \cline{2-3}
     & end & Datum und Uhrzeit am Ende der Messung: \texttt{
co6\_data\_decimal\_6\_3.datetimeto} \\ \hline
\end{longtable}


\subsection{Spontane Atemfrequenz Beatmet} 
\noindent Version 1.0.0

\noindent Profile: \url{https://www.medizininformatik-initiative.de/Kerndatensatz/Modul_Intensivmedizin/Spontane-Atemfrequenz-BeatmetObservation.html}

\noindent Input:
\begin{itemize}
	\item Datensatz aus \texttt{co6\_config\_variables}, \texttt{co6\_medic\_patient}, \\ \texttt{
co6\_data\_decimal\_6\_3}
\end{itemize}
Output:
\begin{itemize}
        \item \ac{fhir}-Ressource der Kategorie \glqq Observation\grqq{}
\end{itemize}
\begin{longtable}{|l|l|p{7.5cm}|}
        \hline
        \rowcolor{lightgray} \multicolumn{3}{|l|}{Data Mapping (inhaltlich)} \\ \hline
        \textbf{id} &  & \texttt{co6\_data\_decimal\_6\_3.id} \\ \hline
	meta & profile & https://medizininformatik-initiative.de/fhir/ext/modul-icu/StructureDefinition/\textbf{
Spontane-Atemfrequenz-Beatmet} \\ \hline 
	status &  & in Abstimmung  \\ \hline 
	category & coding & system: \url{http://snomed.info/sct} \\
\cline{3-3}
	& & code: 40617009 \\ \hline
	code & coding & system: Url von \ac{loinc}, \ac{snomedct}, und / oder \ac{ieee} \\ 
	\cline{3-3} 
	 &  & code: \ac{loinc}, \ac{snomedct}, oder \ac{ieee} \\ \hline
	 \textbf{subject}  & reference & Pseudonymisierte Patientennummer: \texttt{co6\_medic\_patient.patid} \\ \hline
	 \textbf{valueQuantity}  & value & Wert der Messung: \texttt{
co6\_data\_decimal\_6\_3.val} \\
        \cline{2-3}
         & system & http://unitsofmeasure.org \\
         \cline{2-3}
         & code & /min \\ \hline
     \textbf{effectiveDateTime}  & start & Datum und Uhrzeit am Anfang der Messung: \texttt{
co6\_data\_decimal\_6\_3.datetimeto} \\
    \cline{2-3}
     & end & Datum und Uhrzeit am Ende der Messung: \texttt{
co6\_data\_decimal\_6\_3.datetimeto} \\ \hline
\end{longtable}


\subsection{Spontane Mechanische Atemfrequenz Beatmet} 
\noindent Version 1.0.0

\noindent Profile: \url{https://www.medizininformatik-initiative.de/Kerndatensatz/Modul_Intensivmedizin/Spontane-Mechanische-Atemfrequenz-BeatmetObservation.html}

\noindent Input:
\begin{itemize}
	\item Datensatz aus \texttt{co6\_config\_variables}, \texttt{co6\_medic\_patient}, \\ \texttt{
co6\_data\_decimal\_6\_3}
\end{itemize}
Output:
\begin{itemize}
        \item \ac{fhir}-Ressource der Kategorie \glqq Observation\grqq{}
\end{itemize}
\begin{longtable}{|l|l|p{7.5cm}|}
        \hline
        \rowcolor{lightgray} \multicolumn{3}{|l|}{Data Mapping (inhaltlich)} \\ \hline
        \textbf{id} &  & \texttt{co6\_data\_decimal\_6\_3.id} \\ \hline
	meta & profile & https://medizininformatik-initiative.de/fhir/ext/modul-icu/StructureDefinition/\textbf{
Spontane-Mechanische-Atemfrequenz-Beatmet} \\ \hline 
	status &  & in Abstimmung  \\ \hline 
	category & coding & system: \url{http://snomed.info/sct} \\
\cline{3-3}
	& & code: 40617009 \\ \hline
	code & coding & system: Url von \ac{loinc}, \ac{snomedct}, und / oder \ac{ieee} \\ 
	\cline{3-3} 
	 &  & code: \ac{loinc}, \ac{snomedct}, oder \ac{ieee} \\ \hline
	 \textbf{subject}  & reference & Pseudonymisierte Patientennummer: \texttt{co6\_medic\_patient.patid} \\ \hline
	 \textbf{valueQuantity}  & value & Wert der Messung: \texttt{
co6\_data\_decimal\_6\_3.val} \\
        \cline{2-3}
         & system & http://unitsofmeasure.org \\
         \cline{2-3}
         & code & /min \\ \hline
     \textbf{effectiveDateTime}  & start & Datum und Uhrzeit am Anfang der Messung: \texttt{
co6\_data\_decimal\_6\_3.datetimeto} \\
    \cline{2-3}
     & end & Datum und Uhrzeit am Ende der Messung: \texttt{
co6\_data\_decimal\_6\_3.datetimeto} \\ \hline
\end{longtable}


\subsection{Substituatfluss} 
\noindent Version 1.0.0

\noindent Profile: \url{https://www.medizininformatik-initiative.de/Kerndatensatz/Modul_Intensivmedizin/SubstituatflussObservation.html}

\noindent Input:
\begin{itemize}
	\item Datensatz aus \texttt{co6\_config\_variables}, \texttt{co6\_medic\_patient}, \\ \texttt{
co6\_data\_decimal\_6\_3}
\end{itemize}
Output:
\begin{itemize}
        \item \ac{fhir}-Ressource der Kategorie \glqq Observation\grqq{}
\end{itemize}
\begin{longtable}{|l|l|p{7.5cm}|}
        \hline
        \rowcolor{lightgray} \multicolumn{3}{|l|}{Data Mapping (inhaltlich)} \\ \hline
        \textbf{id} &  & \texttt{co6\_data\_decimal\_6\_3.id} \\ \hline
	meta & profile & https://medizininformatik-initiative.de/fhir/ext/modul-icu/StructureDefinition/\textbf{
Substituatfluss} \\ \hline 
	status &  & in Abstimmung  \\ \hline 
	category & coding & system: \url{http://snomed.info/sct} \\
\cline{3-3}
	& & code: 182744004 \\ \hline
	code & coding & system: Url von \ac{loinc}, \ac{snomedct}, und / oder \ac{ieee} \\ 
	\cline{3-3} 
	 &  & code: \ac{loinc}, \ac{snomedct}, oder \ac{ieee} \\ \hline
	 \textbf{subject}  & reference & Pseudonymisierte Patientennummer: \texttt{co6\_medic\_patient.patid} \\ \hline
	 \textbf{valueQuantity}  & value & Wert der Messung: \texttt{
co6\_data\_decimal\_6\_3.val} \\
        \cline{2-3}
         & system & http://unitsofmeasure.org \\
         \cline{2-3}
         & code & mL/h \\ \hline
     \textbf{effectiveDateTime}  & start & Datum und Uhrzeit am Anfang der Messung: \texttt{
co6\_data\_decimal\_6\_3.datetimeto} \\
    \cline{2-3}
     & end & Datum und Uhrzeit am Ende der Messung: \texttt{
co6\_data\_decimal\_6\_3.datetimeto} \\ \hline
\end{longtable}


\subsection{Venöser Druck} 
\noindent Version 1.0.0

\noindent Profile: \url{https://www.medizininformatik-initiative.de/Kerndatensatz/Modul_Intensivmedizin/VenserDruckObservation.html}

\noindent Input:
\begin{itemize}
	\item Datensatz aus \texttt{co6\_config\_variables}, \texttt{co6\_medic\_patient}, \\ \texttt{
co6\_data\_decimal\_6\_3}
\end{itemize}
Output:
\begin{itemize}
        \item \ac{fhir}-Ressource der Kategorie \glqq Observation\grqq{}
\end{itemize}
\begin{longtable}{|l|l|p{7.5cm}|}
        \hline
        \rowcolor{lightgray} \multicolumn{3}{|l|}{Data Mapping (inhaltlich)} \\ \hline
        \textbf{id} &  & \texttt{co6\_data\_decimal\_6\_3.id} \\ \hline
	meta & profile & https://medizininformatik-initiative.de/fhir/ext/modul-icu/StructureDefinition/\textbf{
Venoeser-Druck} \\ \hline 
	status &  & in Abstimmung  \\ \hline 
	category & coding & system: \url{http://snomed.info/sct} \\
\cline{3-3}
	& & code: 182744004 \\ \hline
	code & coding & system: Url von \ac{loinc}, \ac{snomedct}, und / oder \ac{ieee} \\ 
	\cline{3-3} 
	 &  & code: \ac{loinc}, \ac{snomedct}, oder \ac{ieee} \\ \hline
	 \textbf{subject}  & reference & Pseudonymisierte Patientennummer: \texttt{co6\_medic\_patient.patid} \\ \hline
	 \textbf{valueQuantity}  & value & Wert der Messung: \texttt{
co6\_data\_decimal\_6\_3.val} \\
        \cline{2-3}
         & system & http://unitsofmeasure.org \\
         \cline{2-3}
         & code & mm[Hg] \\ \hline
     \textbf{effectiveDateTime}  & start & Datum und Uhrzeit am Anfang der Messung: \texttt{
co6\_data\_decimal\_6\_3.datetimeto} \\
    \cline{2-3}
     & end & Datum und Uhrzeit am Ende der Messung: \texttt{
co6\_data\_decimal\_6\_3.datetimeto} \\ \hline
\end{longtable}


\subsection{Zeitverhältnis Ein-Ausatmung} 
\noindent Version 1.0.0

\noindent Profile: \url{https://www.medizininformatik-initiative.de/Kerndatensatz/Modul_Intensivmedizin/Zeitverhltnis-Ein-AusatmungObservation.html}

\noindent Input:
\begin{itemize}
	\item Datensatz aus \texttt{co6\_config\_variables}, \texttt{co6\_medic\_patient}, \\ \texttt{
co6\_data\_string}
\end{itemize}
Output:
\begin{itemize}
        \item \ac{fhir}-Ressource der Kategorie \glqq Observation\grqq{}
\end{itemize}
\begin{longtable}{|l|l|p{7.5cm}|}
        \hline
        \rowcolor{lightgray} \multicolumn{3}{|l|}{Data Mapping (inhaltlich)} \\ \hline
        \textbf{id} &  & \texttt{co6\_data\_string.id} \\ \hline
	meta & profile & https://medizininformatik-initiative.de/fhir/ext/modul-icu/StructureDefinition/\textbf{
Zeitverhaeltnis-Ein-Ausatmung} \\ \hline 
	status &  & in Abstimmung  \\ \hline 
	category & coding & system: \url{http://snomed.info/sct} \\
\cline{3-3}
	& & code: 40617009 \\ \hline
	code & coding & system: Url von \ac{loinc}, \ac{snomedct}, und / oder \ac{ieee} \\ 
	\cline{3-3} 
	 &  & code: \ac{loinc}, \ac{snomedct}, oder \ac{ieee} \\ \hline
	 \textbf{subject}  & reference & Pseudonymisierte Patientennummer: \texttt{co6\_medic\_patient.patid} \\ \hline
	 \textbf{valueQuantity}  & value & Wert der Messung: \texttt{
co6\_data\_string.val} \\
        \cline{2-3}
         & system & http://unitsofmeasure.org \\
         \cline{2-3}
         & code & {ratio} \\ \hline
     \textbf{effectiveDateTime}  & start & Datum und Uhrzeit am Anfang der Messung: \texttt{
co6\_data\_string.datetimeto} \\
    \cline{2-3}
     & end & Datum und Uhrzeit am Ende der Messung: \texttt{
co6\_data\_string.datetimeto} \\ \hline
\end{longtable}
\clearpage

\subsection{Zentralvenöser Druck (ZVD)} 
\noindent Version 1.0.0

\noindent Profile: \url{https://www.medizininformatik-initiative.de/Kerndatensatz/Modul_Intensivmedizin/ZentralvenserDruckZVDObservation.html}

\noindent Input:
\begin{itemize}
	\item Datensatz aus \texttt{co6\_config\_variables}, \texttt{co6\_medic\_patient}, \\ \texttt{
co6\_data\_decimal\_6\_3}
\end{itemize}
Output:
\begin{itemize}
        \item \ac{fhir}-Ressource der Kategorie \glqq Observation\grqq{}
\end{itemize}
\begin{longtable}{|l|l|p{7.5cm}|}
        \hline
        \rowcolor{lightgray} \multicolumn{3}{|l|}{Data Mapping (inhaltlich)} \\ \hline
        \textbf{id} &  & \texttt{co6\_data\_decimal\_6\_3.id} \\ \hline
	meta & profile & https://medizininformatik-initiative.de/fhir/ext/modul-icu/StructureDefinition/\textbf{
Zentralvenoeser-Druck-(ZVD)} \\ \hline 
	status &  & in Abstimmung  \\ \hline 
	category & coding & system: \url{http://terminology.hl7.org/CodeSystem/observation-category} \\
\cline{3-3}
	& & code: vital-signs \\ \hline
	code & coding & system: Url von \ac{loinc}, \ac{snomedct}, und / oder \ac{ieee} \\ 
	\cline{3-3} 
	 &  & code: \ac{loinc}, \ac{snomedct}, oder \ac{ieee} \\ \hline
	 \textbf{subject}  & reference & Pseudonymisierte Patientennummer: \texttt{co6\_medic\_patient.patid} \\ \hline
	 \textbf{valueQuantity}  & value & Wert der Messung: \texttt{co6\_data\_decimal\_6\_3.val} \\
        \cline{2-3}
         & system & http://unitsofmeasure.org \\
         \cline{2-3}
         & code & mm[Hg] \\ \hline
     \textbf{effectiveDateTime}  & start & Datum und Uhrzeit am Anfang der Messung: \texttt{
co6\_data\_decimal\_6\_3.datetimeto} \\
    \cline{2-3}
     & end & Datum und Uhrzeit am Ende der Messung: \texttt{
co6\_data\_decimal\_6\_3.datetimeto} \\ \hline
\end{longtable}
