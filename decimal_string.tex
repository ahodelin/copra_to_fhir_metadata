\section{Profile} \label{sec:profil}

\subsection{Arterieller Druck} 

Profil: \url{https://www.medizininformatik-initiative.de/Kerndatensatz/Modul_Intensivmedizin/ArteriellerDruckObservation.html}

\begin{longtable}{|l|l|p{7.5cm}|}
        \hline
        \rowcolor{lightgray} \multicolumn{3}{|l|}{Data Mapping (inhaltlich)} \\ \hline
        \textbf{id} &  & \texttt{co6\_data\_decimal\_6\_3.id} \\ \hline
	meta & profile & https://www.medizininformatik-initiative.de/fhir/ext/modul-icu/StructureDefinition/arterieller-druck \\ \hline 
	status &  & final   \\ \hline 
	category & coding & system: \url{http://snomed.info/sct} \\
\cline{3-3}
	& & code: 182744004 \\ \hline
	code & coding & system: Url von \ac{loinc}, \ac{snomedct}, und / oder \ac{ieee} \\ 
	\cline{3-3} 
	 &  & code: \ac{loinc}, \ac{snomedct}, oder \ac{ieee} \\ \hline
	 \textbf{subject} & reference & Pseudonymisierte Patientennummer: \texttt{co6\_data\_string.val} wobei \texttt{co6\_data\_decimal\_6\_3.varID} = 1 und \texttt{co6\_data\_string.varID} = 8 \\ \hline
	 \textbf{valueQuantity}  & \textbf{value} & Wert der Messung: \texttt{
co6\_data\_decimal\_6\_3.val} \\
        \cline{2-3}
         & system & http://unitsofmeasure.org \\
         \cline{2-3}
         & \textbf{code} &
mm[Hg] - \texttt{co6\_config\_variables.unit}
\\ \hline
     \textbf{effectiveDateTime}  &  & Datum und Uhrzeit der Messung: \texttt{
co6\_data\_decimal\_6\_3.datetimeto} \\ \hline
\end{longtable}

\subsection{Atemfrequenz} 

Profil: \url{https://www.medizininformatik-initiative.de/Kerndatensatz/Modul_Intensivmedizin/AtemfrequenzObservation.html}

\begin{longtable}{|l|l|p{7.5cm}|}
        \hline
        \rowcolor{lightgray} \multicolumn{3}{|l|}{Data Mapping (inhaltlich)} \\ \hline
        \textbf{id} &  & \texttt{co6\_data\_decimal\_6\_3.id} \\ \hline
	meta & profile & https://www.medizininformatik.de/fhir/ ext/modul-icu/StructureDefinition/ atemfrequenz \\ \hline 
	status &  & final   \\ \hline 
	category & coding & system: \url{http://terminology.hl7.org/CodeSystem/observation-category} \\
\cline{3-3}
	& & code: vital-signs\\ \hline
	code & coding & system: Url von \ac{loinc}, \ac{snomedct}, und / oder \ac{ieee} \\ 
	\cline{3-3} 
	 &  & code: \ac{loinc}, \ac{snomedct}, oder \ac{ieee} \\ \hline
	 \textbf{subject} & reference & Pseudonymisierte Patientennummer: \texttt{co6\_data\_string.val} wobei \texttt{co6\_data\_decimal\_6\_3.varID} = 1 und \texttt{co6\_data\_string.varID} = 8 \\ \hline
	 \textbf{valueQuantity}  & \textbf{value} & Wert der Messung: \texttt{
co6\_data\_decimal\_6\_3.val} \\
        \cline{2-3}
         & system & http://unitsofmeasure.org \\
         \cline{2-3}
         & \textbf{code} & /min - Transformation notwendig - \texttt{co6\_config\_variables.unit} = bpm 
\\ \hline
     \textbf{effectiveDateTime}  & & Datum und Uhrzeit der Messung: \texttt{
co6\_data\_decimal\_6\_3.datetimeto} \\ \hline
\end{longtable}

\subsection{Atemzugvolumen-Einstellung} 

Profil: \url{https://www.medizininformatik-initiative.de/Kerndatensatz/Modul_Intensivmedizin/Atemzugvolumen-EinstellungObservation.html}

\begin{longtable}{|l|l|p{7.5cm}|}
        \hline
        \rowcolor{lightgray} \multicolumn{3}{|l|}{Data Mapping (inhaltlich)} \\ \hline
        \textbf{id} &  & \texttt{co6\_data\_decimal\_6\_3.id} \\ \hline
	meta & profile & https://www.medizininformatik-initiative.de/fhir/ext/modul-icu/StructureDefinition/atemzugvolumen-einstellung \\ \hline 
	status &  & final   \\ \hline 
	category & coding & system: \url{http://snomed.info/sct} \\
\cline{3-3}
	& & code: 40617009 \\ \hline
	code & coding & system: Url von \ac{loinc}, \ac{snomedct}, und / oder \ac{ieee} \\ 
	\cline{3-3} 
	 &  & code: \ac{loinc}, \ac{snomedct}, oder \ac{ieee} \\ \hline
	 \textbf{subject} & reference & Pseudonymisierte Patientennummer: \texttt{co6\_data\_string.val} wobei \texttt{co6\_data\_decimal\_6\_3.varID} = 1 und \texttt{co6\_data\_string.varID} = 8 \\ \hline
	 \textbf{valueQuantity}  & \textbf{value} & Wert der Messung: \texttt{
co6\_data\_decimal\_6\_3.val} \\
        \cline{2-3}
         & system & http://unitsofmeasure.org \\
         \cline{2-3}
         & \textbf{code} & mL - \texttt{co6\_config\_variables.unit}
\\ \hline
     \textbf{effectiveDateTime}  & & Datum und Uhrzeit der Messung: \texttt{
co6\_data\_decimal\_6\_3.datetimeto} \\
     \hline
\end{longtable}


\subsection{Beatmungsvolumen-Pro-Minute-Machineller-Beatmung} 

Profil: \url{https://www.medizininformatik-initiative.de/Kerndatensatz/Modul_Intensivmedizin/Beatmungsvolumen-Pro-Minute-Machineller-BeatmungObservation.html}

\begin{longtable}{|l|l|p{7.5cm}|}
        \hline
        \rowcolor{lightgray} \multicolumn{3}{|l|}{Data Mapping (inhaltlich)} \\ \hline
        \textbf{id} &  & \texttt{co6\_data\_decimal\_6\_3.id} \\ \hline
	meta & profile & https://www.medizininformatik-initiative.de/fhir/ext/modul-icu/StructureDefinition/ beatmungsvolumen-pro-minute-maschineller-beatmung \\ \hline 
	status &  & final   \\ \hline 
	category & coding & system: \url{http://snomed.info/sct} \\
\cline{3-3}
	& & code: 40617009 \\ \hline
	code & coding & system: Url von \ac{loinc}, \ac{snomedct}, und / oder \ac{ieee} \\ 
	\cline{3-3} 
	 &  & code: \ac{loinc}, \ac{snomedct}, oder \ac{ieee} \\ \hline
	\textbf{subject} & reference & Pseudonymisierte Patientennummer: \texttt{co6\_data\_string.val} wobei \texttt{co6\_data\_decimal\_6\_3.varID} = 1 und \texttt{co6\_data\_string.varID} = 8 \\ \hline
	 \textbf{valueQuantity}  & \textbf{value} & Wert der Messung: \texttt{
co6\_data\_decimal\_6\_3.val} \\
        \cline{2-3}
         & system & http://unitsofmeasure.org \\
         \cline{2-3}
         & \textbf{code} &
L/min - \texttt{co6\_config\_variables.unit}
\\ \hline
     \textbf{effectiveDateTime}  & & Datum und Uhrzeit der Messung: \texttt{
co6\_data\_decimal\_6\_3.datetimeto} \\
     \hline
\end{longtable}

\subsection{Beatmungsvolumenzeit auf hohem Druck} 
Profil: \url{https://www.medizininformatik-initiative.de/Kerndatensatz/Modul_Intensivmedizin/BeatmungszeitaufhohemDruckObservation.html}

\begin{longtable}{|l|l|p{7.5cm}|}
	\hline
	\rowcolor{lightgray} \multicolumn{3}{|l|}{Data Mapping (inhaltlich)} \\ \hline
	\textbf{id} &  & \texttt{co6\_data\_decimal\_6\_3.id} \\ \hline
	meta & profile & https://www.medizininformatik-initiative.de/fhir/ext/modul-icu/StructureDefinition/beatmungszeit-hohem-druck \\ \hline 
	status &  & final   \\ \hline 
	category & coding & system: \url{http://snomed.info/sct} \\
	\cline{3-3}
	& & code: 40617009 \\ \hline
	code & coding & system: Url von \ac{loinc}, \ac{snomedct}, und / oder \ac{ieee} \\ 
	\cline{3-3} 
	&  & code: \ac{loinc}, \ac{snomedct}, oder \ac{ieee} \\ \hline
	\textbf{subject} & reference & Pseudonymisierte Patientennummer: \texttt{co6\_data\_string.val} wobei \texttt{co6\_data\_decimal\_6\_3.varID} = 1 und \texttt{co6\_data\_string.varID} = 8 \\ \hline
	\textbf{valueQuantity}  & \textbf{value} & Wert der Messung: \texttt{
		co6\_data\_decimal\_6\_3.val} \\
	\cline{2-3}
	& system & http://unitsofmeasure.org \\
	\cline{2-3}
	& \textbf{code} & s - \texttt{co6\_config\_variables.unit}
	\\ \hline
	\textbf{effectiveDateTime}  & & Datum und Uhrzeit der Messung: \texttt{
		co6\_data\_decimal\_6\_3.datetimeto} \\
	\hline
\end{longtable}

\subsection{Beatmungsvolumenzeit auf niedrigem Druck} 
Profil: \url{https://www.medizininformatik-initiative.de/Kerndatensatz/Modul_Intensivmedizin/BeatmungszeitaufniedrigemDruckObservation.html}

\begin{longtable}{|l|l|p{7.5cm}|}
	\hline
	\rowcolor{lightgray} \multicolumn{3}{|l|}{Data Mapping (inhaltlich)} \\ \hline
	\textbf{id} &  & \texttt{co6\_data\_decimal\_6\_3.id} \\ \hline
	meta & profile & https://www.medizininformatik-initiative.de/fhir/ext/modul-icu/StructureDefinition/beatmungszeit-niedrigem-druck \\ \hline 
	status &  & final   \\ \hline 
	category & coding & system: \url{http://snomed.info/sct} \\
	\cline{3-3}
	& & code: 40617009 \\ \hline
	code & coding & system: Url von \ac{loinc}, \ac{snomedct}, und / oder \ac{ieee} \\ 
	\cline{3-3} 
	&  & code: \ac{loinc}, \ac{snomedct}, oder \ac{ieee} \\ \hline
	\textbf{subject} & reference & Pseudonymisierte Patientennummer: \texttt{co6\_data\_string.val} wobei \texttt{co6\_data\_decimal\_6\_3.varID} = 1 und \texttt{co6\_data\_string.varID} = 8 \\ \hline
	\textbf{valueQuantity}  & \textbf{value} & Wert der Messung: \texttt{
		co6\_data\_decimal\_6\_3.val} \\
	\cline{2-3}
	& system & http://unitsofmeasure.org \\
	\cline{2-3}
	& \textbf{code} & s - \texttt{co6\_config\_variables.unit}
	\\ \hline
	\textbf{effectiveDateTime}  & & Datum und Uhrzeit der Messung: \texttt{
		co6\_data\_decimal\_6\_3.datetimeto} \\
	\hline
\end{longtable}


\subsection{Blutfluss durch cardiovasculäres Gerät} 

Profil: \url{https://www.medizininformatik-initiative.de/Kerndatensatz/Modul_Intensivmedizin/BlutflussdurchcardiovasculresGertObservation.html}

\begin{longtable}{|l|l|p{7.5cm}|}
        \hline
        \rowcolor{lightgray} \multicolumn{3}{|l|}{Data Mapping (inhaltlich)} \\ \hline
        \textbf{id} &  & \texttt{co6\_data\_decimal\_6\_3.id} \\ \hline
	meta & profile & https://www.medizininformatik-initiative.de/fhir/ext/modul-icu/StructureDefinition/blutfluss-cardiovasculaeres-geraet \\ \hline 
	status &  & final   \\ \hline 
	category & coding & system: \url{http://snomed.info/sct} \\
\cline{3-3}
	& & code: 182744004 \\ \hline
	code & coding & system: Url von \ac{loinc}, \ac{snomedct}, und / oder \ac{ieee} \\ 
	\cline{3-3} 
	 &  & code: \ac{loinc}, \ac{snomedct}, oder \ac{ieee} \\ \hline
	 \textbf{subject} & reference & Pseudonymisierte Patientennummer: \texttt{co6\_data\_string.val} wobei \texttt{co6\_data\_decimal\_6\_3.varID} = 1 und \texttt{co6\_data\_string.varID} = 8 \\ \hline
	 \textbf{valueQuantity}  & \textbf{value} & Wert der Messung: \texttt{
co6\_data\_decimal\_6\_3.val} \\
        \cline{2-3}
         & system & http://unitsofmeasure.org \\
         \cline{2-3}
         & \textbf{code} & L/min - \texttt{co6\_config\_variables.unit}
\\ \hline
     \textbf{effectiveDateTime}  & & Datum und Uhrzeit der Messung: \texttt{co6\_data\_decimal\_6\_3.datetimeto} \\
     \hline
\end{longtable}


\subsection{Blutfluss extrakorporaler Gassaustausch} 

Profil: \url{https://www.medizininformatik-initiative.de/Kerndatensatz/Modul_Intensivmedizin/BlutflussextrakorporalerGasaustauschObservation.html}

\begin{longtable}{|l|l|p{7.5cm}|}
	\hline
	\rowcolor{lightgray} \multicolumn{3}{|l|}{Data Mapping (inhaltlich)} \\ \hline
	\textbf{id} &  & \texttt{co6\_data\_decimal\_6\_3.id} \\ \hline
	meta & profile & https://www.medizininformatik-initiative.de/fhir/ext/modul-icu/StructureDefinition/blutfluss-extrakorporaler-gasaustausch \\ \hline 
	status &  & final   \\ \hline 
	category & coding & system: \url{http://snomed.info/sct} \\
	\cline{3-3}
	& & code: 182744004 \\ \hline
	code & coding & system: Url von \ac{loinc}, \ac{snomedct}, und / oder \ac{ieee} \\ 
	\cline{3-3} 
	&  & code: \ac{loinc}, \ac{snomedct}, oder \ac{ieee} \\ \hline
	\textbf{subject} & reference & Pseudonymisierte Patientennummer: \texttt{co6\_data\_string.val} wobei \texttt{co6\_data\_decimal\_6\_3.varID} = 1 und \texttt{co6\_data\_string.varID} = 8 \\ \hline
	\textbf{valueQuantity}  & \textbf{value} & Wert der Messung: \texttt{
		co6\_data\_decimal\_6\_3.val} \\
	\cline{2-3}
	& system & http://unitsofmeasure.org \\
	\cline{2-3}
	& \textbf{code} & L/min - \texttt{co6\_config\_variables.unit}
	\\ \hline
	\textbf{effectiveDateTime}  & & Datum und Uhrzeit der Messung: \texttt{co6\_data\_decimal\_6\_3.datetimeto} \\
	\hline
\end{longtable}

\subsection{Dauer Hämodialysesitzung} 

Profil: \url{https://www.medizininformatik-initiative.de/Kerndatensatz/Modul_Intensivmedizin/DauerHmodialysesitzungObservation.html}

\begin{longtable}{|l|l|p{7.5cm}|}
	\hline
	\rowcolor{lightgray} \multicolumn{3}{|l|}{Data Mapping (inhaltlich)} \\ \hline
	\textbf{id} &  & \texttt{co6\_data\_decimal\_6\_3.id} \\ \hline
	meta & profile & https://www.medizininformatik-initiative.de/fhir/ext/modul-icu/StructureDefinition/dauer-haemodialysesitzung \\ \hline 
	status &  & final   \\ \hline 
	category & coding & system: \url{http://snomed.info/sct} \\
	\cline{3-3}
	& & code: 182744004 \\ \hline
	code & coding & system: Url von \ac{loinc}, \ac{snomedct}, und / oder \ac{ieee} \\ 
	\cline{3-3} 
	&  & code: \ac{loinc}, \ac{snomedct}, oder \ac{ieee} \\ \hline
	\textbf{subject} & reference & Pseudonymisierte Patientennummer: \texttt{co6\_data\_string.val} wobei \texttt{co6\_data\_decimal\_6\_3.varID} = 1 und \texttt{co6\_data\_string.varID} = 8 \\ \hline
	\textbf{valueQuantity}  & \textbf{value} & Wert der Messung: \texttt{
		co6\_data\_decimal\_6\_3.val} \\
	\cline{2-3}
	& system & http://unitsofmeasure.org \\
	\cline{2-3}
	& \textbf{code} & h - Transformation notwendig - \texttt{co6\_config\_variables.unit} = h:min
	\\ \hline
	\textbf{effectiveDateTime}  & & Datum und Uhrzeit der Messung: \texttt{co6\_data\_decimal\_6\_3.datetimeto} \\
	\hline
\end{longtable}

\subsection{Druckdifferenz Beatmung} 

 Profil: \url{https://www.medizininformatik-initiative.de/Kerndatensatz/Modul_Intensivmedizin/DruckdifferenzBeatmungObservation.html}

\begin{longtable}{|l|l|p{7.5cm}|}
        \hline
        \rowcolor{lightgray} \multicolumn{3}{|l|}{Data Mapping (inhaltlich)} \\ \hline
        \textbf{id} &  & \texttt{co6\_data\_decimal\_6\_3.id} \\ \hline
	meta & profile & https://www.medizininformatik-initiative.de/fhir/ext/modul-icu/StructureDefinition/druckdifferenz-beatmung \\ \hline 
	status &  & final   \\ \hline 
	category & coding & system: \url{http://snomed.info/sct} \\
\cline{3-3}
	& & code: 40617009 \\ \hline
	code & coding & system: Url von \ac{loinc}, \ac{snomedct}, und / oder \ac{ieee} \\ 
	\cline{3-3} 
	 &  & code: \ac{loinc}, \ac{snomedct}, oder \ac{ieee} \\ \hline
	 \textbf{subject} & reference & Pseudonymisierte Patientennummer: \texttt{co6\_data\_string.val} wobei \texttt{co6\_data\_decimal\_6\_3.varID} = 1 und \texttt{co6\_data\_string.varID} = 8 \\ \hline
	 \textbf{valueQuantity}  & \textbf{value} & Wert der Messung: \texttt{
co6\_data\_decimal\_6\_3.val} \\
        \cline{2-3}
         & system & http://unitsofmeasure.org \\
         \cline{2-3}
         & \textbf{code} &
cm[H2O] - \texttt{co6\_config\_variables.unit}
\\ \hline
     \textbf{effectiveDateTime}  & & Datum und Uhrzeit der Messung: \texttt{
co6\_data\_decimal\_6\_3.datetimeto} \\ \hline
\end{longtable}

\subsection{Dynamische Kompliance} 

Profil: \url{https://www.medizininformatik-initiative.de/Kerndatensatz/Modul_Intensivmedizin/DynamischeKomplianceObservation.html}

\begin{longtable}{|l|l|p{7.5cm}|}
	\hline
	\rowcolor{lightgray} \multicolumn{3}{|l|}{Data Mapping (inhaltlich)} \\ \hline
	\textbf{id} &  & \texttt{co6\_data\_decimal\_6\_3.id} \\ \hline
	meta & profile & https://www.medizininformatik-initiative.de/fhir/ext/modul-icu/StructureDefinition/dynamische-kompliance \\ \hline 
	status &  & final   \\ \hline 
	category & coding & system: \url{http://snomed.info/sct} \\
	\cline{3-3}
	& & code: 40617009 \\ \hline
	code & coding & system: Url von \ac{loinc}, \ac{snomedct}, und / oder \ac{ieee} \\ 
	\cline{3-3} 
	&  & code: \ac{loinc}, \ac{snomedct}, oder \ac{ieee} \\ \hline
	\textbf{subject} & reference & Pseudonymisierte Patientennummer: \texttt{co6\_data\_string.val} wobei \texttt{co6\_data\_decimal\_6\_3.varID} = 1 und \texttt{co6\_data\_string.varID} = 8 \\ \hline
	\textbf{valueQuantity}  & \textbf{value} & Wert der Messung: \texttt{
		co6\_data\_decimal\_6\_3.val} \\
	\cline{2-3}
	& system & http://unitsofmeasure.org \\
	\cline{2-3}
	& \textbf{code} &
	ml/cm[H2O] - Transformation notwendig - \texttt{co6\_config\_variables.unit} = ml/mbar
	\\ \hline
	\textbf{effectiveDateTime}  & & Datum und Uhrzeit der Messung: \texttt{
		co6\_data\_decimal\_6\_3.datetimeto} \\ \hline
\end{longtable}

\subsection{Eingestellter inspiratorischer Gasfluss} 

Profil: \url{https://www.medizininformatik-initiative.de/Kerndatensatz/Modul_Intensivmedizin/EingestellterinspiratorischerGasflussObservation.html}

\begin{longtable}{|l|l|p{7.5cm}|}
	\hline
	\rowcolor{lightgray} \multicolumn{3}{|l|}{Data Mapping (inhaltlich)} \\ \hline
	\textbf{id} &  & \texttt{co6\_data\_decimal\_6\_3.id} \\ \hline
	meta & profile & https://www.medizininformatik-initiative.de/fhir/ext/modul-icu/StructureDefinition/eingestellter-inspiratorischer-gasfluss \\ \hline 
	status &  & final   \\ \hline 
	category & coding & system: \url{http://snomed.info/sct} \\
	\cline{3-3}
	& & code: 40617009 \\ \hline
	code & coding & system: Url von \ac{loinc}, \ac{snomedct}, und / oder \ac{ieee} \\ 
	\cline{3-3} 
	&  & code: \ac{loinc}, \ac{snomedct}, oder \ac{ieee} \\ \hline
	\textbf{subject} & reference & Pseudonymisierte Patientennummer: \texttt{co6\_data\_string.val} wobei \texttt{co6\_data\_decimal\_6\_3.varID} = 1 und \texttt{co6\_data\_string.varID} = 8 \\ \hline
	\textbf{valueQuantity}  & \textbf{value} & Wert der Messung: \texttt{
		co6\_data\_decimal\_6\_3.val} \\
	\cline{2-3}
	& system & http://unitsofmeasure.org \\
	\cline{2-3}
	& \textbf{code} &
	L/min - \texttt{co6\_config\_variables.unit}
	\\ \hline
	\textbf{effectiveDateTime}  & & Datum und Uhrzeit der Messung: \texttt{
		co6\_data\_decimal\_6\_3.datetimeto} \\ \hline
\end{longtable}

\subsection{Einstellung-Einatmungszeit-Beatmung}

Profil: \url{https://www.medizininformatik-initiative.de/Kerndatensatz/Modul_Intensivmedizin/Einstellung-Einatmungszeit-BeatmungObservation.html}

\begin{longtable}{|l|l|p{7.5cm}|}
        \hline
        \rowcolor{lightgray} \multicolumn{3}{|l|}{Data Mapping (inhaltlich)} \\ \hline
        \textbf{id} &  & \texttt{co6\_data\_decimal\_6\_3.id} \\ \hline
	meta & profile & https://www.medizininformatik-initiative.de/fhir/ext/modul-icu/StructureDefinition/einstellung-einatmungszeit-beatmung \\ \hline 
	status &  & final   \\ \hline 
	category & coding & system: \url{http://snomed.info/sct} \\
\cline{3-3}
	& & code: 40617009\\ \hline
	code & coding & system: Url von \ac{loinc}, \ac{snomedct}, und / oder \ac{ieee} \\ 
	\cline{3-3} 
	 &  & code: \ac{loinc}, \ac{snomedct}, oder \ac{ieee} \\ \hline
	 \textbf{subject} & reference & Pseudonymisierte Patientennummer: \texttt{co6\_data\_string.val} wobei \texttt{co6\_data\_decimal\_6\_3.varID} = 1 und \texttt{co6\_data\_string.varID} = 8 \\ \hline
	 \textbf{valueQuantity}  & \textbf{value} & Wert der Messung: \texttt{
co6\_data\_decimal\_6\_3.val} \\
        \cline{2-3}
         & system & http://unitsofmeasure.org \\
         \cline{2-3}
         & \textbf{code} & s - \texttt{co6\_config\_variables.unit} \\ \hline
     \textbf{effectiveDateTime}  & & Datum und Uhrzeit der Messung: \texttt{co6\_data\_decimal\_6\_3.datetimeto} \\ \hline
\end{longtable}

\subsection{Herzfrequenz} 

Profil: \url{https://www.medizininformatik-initiative.de/Kerndatensatz/Modul_Intensivmedizin/HerzfrequenzObservation.html}

\begin{longtable}{|l|l|p{7.5cm}|}
        \hline
        \rowcolor{lightgray} \multicolumn{3}{|l|}{Data Mapping (inhaltlich)} \\ \hline
        \textbf{id} &  & \texttt{co6\_data\_decimal\_6\_3.id} \\ \hline
	meta & profile & https://www.medizininformatik-initiative.de/fhir/ext/modul-icu/StructureDefinition/herzfrequenz \\ \hline 
	status &  & final   \\ \hline 
	category & coding & system: \url{http://terminology.hl7.org/CodeSystem/observation-category} \\
\cline{3-3}
	& & code: vital-signs\\ \hline
	code & coding & system: Url von \ac{loinc}, \ac{snomedct}, und / oder \ac{ieee} \\ 
	\cline{3-3} 
	 &  & code: \ac{loinc}, \ac{snomedct}, oder \ac{ieee} \\ \hline
	 \textbf{subject} & reference & Pseudonymisierte Patientennummer: \texttt{co6\_data\_string.val} wobei \texttt{co6\_data\_decimal\_6\_3.varID} = 1 und \texttt{co6\_data\_string.varID} = 8 \\ \hline
	 \textbf{valueQuantity}  & \textbf{value} & Wert der Messung: \texttt{
co6\_data\_decimal\_6\_3.val} \\
        \cline{2-3}
         & system & http://unitsofmeasure.org \\
         \cline{2-3}
         & \textbf{code} & /min - \texttt{co6\_config\_variables.unit} \\ \hline
     \textbf{effectiveDateTime}  & & Datum und Uhrzeit der Messung: \texttt{
co6\_data\_decimal\_6\_3.datetimeto} \\ \hline
\end{longtable}

\subsection{Herzzeitvolumen} 

Profil: \url{https://www.medizininformatik-initiative.de/Kerndatensatz/Modul_Intensivmedizin/HerzzeitvolumenObservation.html}

\begin{longtable}{|l|l|p{7.5cm}|}
        \hline
        \rowcolor{lightgray} \multicolumn{3}{|l|}{Data Mapping (inhaltlich)} \\ \hline
        \textbf{id} &  & \texttt{co6\_data\_decimal\_6\_3.id} \\ \hline
	meta & profile & https://www.medizininformatik-initiative.de/fhir/ext/modul-icu/StructureDefinition/herzzeitvolumen \\ \hline 
	status &  & final   \\ \hline 
	category & coding & system: \url{http://terminology.hl7.org/CodeSystem/observation-category} \\
\cline{3-3}
	& & code: vital-signs \\ \hline
	code & coding & system: Url von \ac{loinc}, \ac{snomedct}, und / oder \ac{ieee} \\ 
	\cline{3-3} 
	 &  & code: \ac{loinc}, \ac{snomedct}, oder \ac{ieee} \\ \hline
	 \textbf{subject} & reference & Pseudonymisierte Patientennummer: \texttt{co6\_data\_string.val} wobei \texttt{co6\_data\_decimal\_6\_3.varID} = 1 und \texttt{co6\_data\_string.varID} = 8 \\ \hline
	 \textbf{valueQuantity}  & \textbf{value} & Wert der Messung: \texttt{
co6\_data\_decimal\_6\_3.val} \\
        \cline{2-3}
         & system & http://unitsofmeasure.org \\
         \cline{2-3}
         & \textbf{code} &
L/min - \texttt{co6\_config\_variables.unit}
\\ \hline
     \textbf{effectiveDateTime}  & & Datum und Uhrzeit der Messung: \texttt{
co6\_data\_decimal\_6\_3.datetimeto} \\
     \hline
\end{longtable}


\subsection{Inspiratorischer Gasfluss} 

Profil: \url{https://www.medizininformatik-initiative.de/Kerndatensatz/Modul_Intensivmedizin/InspiratorischerGasflussObservation.html}

\begin{longtable}{|l|l|p{7.5cm}|}
        \hline
        \rowcolor{lightgray} \multicolumn{3}{|l|}{Data Mapping (inhaltlich)} \\ \hline
        \textbf{id} &  & \texttt{co6\_data\_decimal\_6\_3.id} \\ \hline
	meta & profile & https://www.medizininformatik-initiative.de/fhir/ext/modul-icu/StructureDefinition/inspiratorischer-gasfluss \\ \hline 
	status &  & final   \\ \hline 
	category & coding & system: \url{http://snomed.info/sct} \\
\cline{3-3}
	& & code: 40617009 \\ \hline
	code & coding & system: Url von \ac{loinc}, \ac{snomedct}, und / oder \ac{ieee} \\ 
	\cline{3-3} 
	 &  & code: \ac{loinc}, \ac{snomedct}, oder \ac{ieee} \\ \hline
	 \textbf{subject} & reference & Pseudonymisierte Patientennummer: \texttt{co6\_data\_string.val} wobei \texttt{co6\_data\_decimal\_6\_3.varID} = 1 und \texttt{co6\_data\_string.varID} = 8 \\ \hline
	 \textbf{valueQuantity}  & \textbf{value} & Wert der Messung: \texttt{
co6\_data\_decimal\_6\_3.val} \\
        \cline{2-3}
         & system & http://unitsofmeasure.org \\
         \cline{2-3}
         & \textbf{code} & L/min - \texttt{co6\_config\_variables.unit} \\ \hline
     \textbf{effectiveDateTime}  & & Datum und Uhrzeit der Messung: \texttt{
co6\_data\_decimal\_6\_3.datetimeto} \\
     \hline
\end{longtable}

\subsection{Inspiratorische Sauerstofffraktion eingestellt} 

Profil: \url{https://www.medizininformatik-initiative.de/Kerndatensatz/Modul_Intensivmedizin/Inspiratorische-Sauerstofffraktion-eingestellt--Observation.html}

\begin{longtable}{|l|l|p{7.5cm}|}
	\hline
	\rowcolor{lightgray} \multicolumn{3}{|l|}{Data Mapping (inhaltlich)} \\ \hline
	\textbf{id} &  & \texttt{co6\_data\_decimal\_6\_3.id} \\ \hline
	meta & profile & https://www.medizininformatik-initiative.de/fhir/ext/modul-icu/StructureDefinition/inspiratorisch-sauerstofffraktion-eingestellt \\ \hline 
	status &  & final   \\ \hline 
	category & coding & system: \url{http://snomed.info/sct} \\
	\cline{3-3}
	& & code: 40617009 \\ \hline
	code & coding & system: Url von \ac{loinc}, \ac{snomedct}, und / oder \ac{ieee} \\ 
	\cline{3-3} 
	&  & code: \ac{loinc}, \ac{snomedct}, oder \ac{ieee} \\ \hline
	\textbf{subject} & reference & Pseudonymisierte Patientennummer: \texttt{co6\_data\_string.val} wobei \texttt{co6\_data\_decimal\_6\_3.varID} = 1 und \texttt{co6\_data\_string.varID} = 8 \\ \hline
	\textbf{valueQuantity}  & \textbf{value} & Wert der Messung: \texttt{
		co6\_data\_decimal\_6\_3.val} \\
	\cline{2-3}
	& system & http://unitsofmeasure.org \\
	\cline{2-3}
	& \textbf{code} & \% - \texttt{co6\_config\_variables.unit}
	\\ \hline
	\textbf{effectiveDateTime}  & & Datum und Uhrzeit der Messung: \texttt{
		co6\_data\_decimal\_6\_3.datetimeto} \\
	\hline
\end{longtable}

\subsection{Inspiratorische Sauerstofffraktion gemessen} 

Profil: \url{https://www.medizininformatik-initiative.de/Kerndatensatz/Modul_Intensivmedizin/Inspiratorische-Sauerstofffraktion-gemessen--Observation-duplicate-2.html}

\begin{longtable}{|l|l|p{7.5cm}|}
	\hline
	\rowcolor{lightgray} \multicolumn{3}{|l|}{Data Mapping (inhaltlich)} \\ \hline
	\textbf{id} &  & \texttt{co6\_data\_decimal\_6\_3.id} \\ \hline
	meta & profile & https://www.medizininformatik-initiative.de/fhir/ext/modul-icu/StructureDefinition/inspiratorische-sauerstofffraktion \\ \hline 
	status &  & final   \\ \hline 
	category & coding & system: \url{http://snomed.info/sct} \\
	\cline{3-3}
	& & code: 40617009 \\ \hline
	code & coding & system: Url von \ac{loinc}, \ac{snomedct}, und / oder \ac{ieee} \\ 
	\cline{3-3} 
	&  & code: \ac{loinc}, \ac{snomedct}, oder \ac{ieee} \\ \hline
	\textbf{subject} & reference & Pseudonymisierte Patientennummer: \texttt{co6\_data\_string.val} wobei \texttt{co6\_data\_decimal\_6\_3.varID} = 1 und \texttt{co6\_data\_string.varID} = 8 \\ \hline
	\textbf{valueQuantity}  & \textbf{value} & Wert der Messung: \texttt{
		co6\_data\_decimal\_6\_3.val} \\
	\cline{2-3}
	& system & http://unitsofmeasure.org \\
	\cline{2-3}
	& \textbf{code} & 1 - Transformation notwendig - \texttt{co6\_data\_decimal\_6\_3.unit} = \%
	\\ \hline
	\textbf{effectiveDateTime}  & & Datum und Uhrzeit der Messung: \texttt{
		co6\_data\_decimal\_6\_3.datetimeto} \\
	\hline
\end{longtable}

\subsection{Intrakranieller Druck (ICP)} 
 Profil: \url{https://www.medizininformatik-initiative.de/Kerndatensatz/Modul_Intensivmedizin/Intrakranieller-Druck-ICP--Observation.html}

\begin{longtable}{|l|l|p{7.5cm}|}
        \hline
        \rowcolor{lightgray} \multicolumn{3}{|l|}{Data Mapping (inhaltlich)} \\ \hline
        \textbf{id} &  & \texttt{co6\_data\_decimal\_6\_3.id} \\ \hline
	meta & profile & https://www.medizininformatik-initiative.de/fhir/ext/modul-icu/StructureDefinition/intrakranieller-druck-icp \\ \hline 
	status &  & final   \\ \hline 
	category & coding & system: \url{http://terminology.hl7.org/CodeSystem/observation-category} \\
\cline{3-3}
	& & code: vital-signs \\ \hline
	code & coding & system: Url von \ac{loinc}, \ac{snomedct}, und / oder \ac{ieee} \\ 
	\cline{3-3} 
	 &  & code: \ac{loinc}, \ac{snomedct}, oder \ac{ieee} \\ \hline
	 \textbf{subject} & reference & Pseudonymisierte Patientennummer: \texttt{co6\_data\_string.val} wobei \texttt{co6\_data\_decimal\_6\_3.varID} = 1 und \texttt{co6\_data\_string.varID} = 8 \\ \hline
	 \textbf{valueQuantity}  & \textbf{value} & Wert der Messung: \texttt{
co6\_data\_decimal\_6\_3.val} \\
        \cline{2-3}
         & system & http://unitsofmeasure.org \\
         \cline{2-3}
         & \textbf{code} & mm[Hg] - \texttt{co6\_config\_variables.unit} \\ \hline
     \textbf{effectiveDateTime}  & & Datum und Uhrzeit der Messung: \texttt{
co6\_data\_decimal\_6\_3.datetimeto} \\
     \hline
\end{longtable}

\subsection{Ionisiertes Kalzium aus Nierenersatzverfahren} 
Profil: \url{https://www.medizininformatik-initiative.de/Kerndatensatz/Modul_Intensivmedizin/IonisiertesKalziumausNierenersatzverfahrenObservation.html}

\begin{longtable}{|l|l|p{7.5cm}|}
        \hline
        \rowcolor{lightgray} \multicolumn{3}{|l|}{Data Mapping (inhaltlich)} \\ \hline
        \textbf{id} &  & \texttt{co6\_data\_decimal\_6\_3.id} \\ \hline
	meta & profile & https://www.medizininformatik-initiative.de/fhir/ext/modul-icu/StructureDefinition/ionisiertes-kalzium-nierenersatzverfahren \\ \hline 
	status &  & final   \\ \hline 
	category & coding & system: \url{http://snomed.info/sct} \\
\cline{3-3}
	& & code: 182744004 \\ \hline
	code & coding & system: Url von \ac{loinc}, \ac{snomedct}, und / oder \ac{ieee} \\ 
	\cline{3-3} 
	 &  & code: \ac{loinc}, \ac{snomedct}, oder \ac{ieee} \\ \hline
	 \textbf{subject} & reference & Pseudonymisierte Patientennummer: \texttt{co6\_data\_string.val} wobei \texttt{co6\_data\_decimal\_6\_3.varID} = 1 und \texttt{co6\_data\_string.varID} = 8 \\ \hline
	 \textbf{valueQuantity}  & \textbf{value} & Wert der Messung: \texttt{
co6\_data\_decimal\_6\_3.val} \\
        \cline{2-3}
         & system & http://unitsofmeasure.org \\
         \cline{2-3}
         & \textbf{code} & mmol/L - \texttt{co6\_config\_variables.unit} \\ \hline
     \textbf{effectiveDateTime}  & & Datum und Uhrzeit der Messung: \texttt{
co6\_data\_decimal\_6\_3.datetimeto} \\ \hline
\end{longtable}

\subsection{Kopfumfang} 

Profil: \url{https://www.medizininformatik-initiative.de/Kerndatensatz/Modul_Intensivmedizin/KopfumfangObservation.html}

\begin{longtable}{|l|l|p{7.5cm}|}
	\hline
	\rowcolor{lightgray} \multicolumn{3}{|l|}{Data Mapping (inhaltlich)} \\ \hline
	\textbf{id} &  & \texttt{co6\_data\_decimal\_6\_3.id} \\ \hline
	meta & profile & https://www.medizininformatik-initiative.de/fhir/ext/modul-icu/StructureDefinition/kopfumfang \\ \hline 
	status &  & final   \\ \hline 
	category & coding & system: \url{http://terminology.hl7.org/CodeSystem/observation-category} \\
	\cline{3-3}
	& & code: vital-signs \\ \hline
	code & coding & system: Url von \ac{loinc}, \ac{snomedct}, und / oder \ac{ieee} \\ 
	\cline{3-3} 
	&  & code: \ac{loinc}, \ac{snomedct}, oder \ac{ieee} \\ \hline
	\textbf{subject} & reference & Pseudonymisierte Patientennummer: \texttt{co6\_data\_string.val} wobei \texttt{co6\_data\_decimal\_6\_3.varID} = 1 und \texttt{co6\_data\_string.varID} = 8 \\ \hline
	\textbf{valueQuantity}  & \textbf{value} & Wert der Messung: \texttt{
		co6\_data\_decimal\_6\_3.val} \\
	\cline{2-3}
	& system & http://unitsofmeasure.org \\
	\cline{2-3}
	& \textbf{code} & cm - \texttt{co6\_config\_variables.unit} \\ \hline
	\textbf{effectiveDateTime}  & & Datum und Uhrzeit der Messung: \texttt{
		co6\_data\_decimal\_6\_3.datetimeto} \\ \hline
\end{longtable}

\subsection{Körpergewicht} 

Profil: \url{https://www.medizininformatik-initiative.de/Kerndatensatz/Modul_Intensivmedizin/KoerpergewichtObservation.html}

\begin{longtable}{|l|l|p{7.5cm}|}
        \hline
        \rowcolor{lightgray} \multicolumn{3}{|l|}{Data Mapping (inhaltlich)} \\ \hline
        \textbf{id} &  & \texttt{co6\_data\_decimal\_6\_3.id} \\ \hline
	meta & profile & https://www.medizininformatik-initiative.de/fhir/ext/modul-icu/StructureDefinition/koerpergewicht \\ \hline 
	status &  & final   \\ \hline 
	category & coding & system: \url{http://terminology.hl7.org/CodeSystem/observation-category} \\
\cline{3-3}
	& & code: vital-signs \\ \hline
	code & coding & system: Url von \ac{loinc}, \ac{snomedct}, und / oder \ac{ieee} \\ 
	\cline{3-3} 
	 &  & code: \ac{loinc}, \ac{snomedct}, oder \ac{ieee} \\ \hline
	\textbf{subject} & reference & Pseudonymisierte Patientennummer: \texttt{co6\_data\_string.val} wobei \texttt{co6\_data\_decimal\_6\_3.varID} = 1 und \texttt{co6\_data\_string.varID} = 8 \\ \hline
	 \textbf{valueQuantity}  & \textbf{value} & Wert der Messung: \texttt{
co6\_data\_decimal\_6\_3.val} \\
        \cline{2-3}
         & system & http://unitsofmeasure.org \\
         \cline{2-3}
         & \textbf{code} & kg - \texttt{co6\_config\_variables.unit} \\ \hline
     \textbf{effectiveDateTime}  & & Datum und Uhrzeit der Messung: \texttt{co6\_data\_decimal\_6\_3.datetimeto} \\ \hline
\end{longtable}

\subsection{Körpergröße} 
Profil: \url{https://www.medizininformatik-initiative.de/Kerndatensatz/Modul_Intensivmedizin/KoerpergroesseObservation.html}

\begin{longtable}{|l|l|p{7.5cm}|}
        \hline
        \rowcolor{lightgray} \multicolumn{3}{|l|}{Data Mapping (inhaltlich)} \\ \hline
        \textbf{id} &  & \texttt{co6\_data\_decimal\_6\_3.id} \\ \hline
	meta & profile & https://www.medizininformatik-initiative.de/fhir/ext/modul-icu/StructureDefinition/koerpergroesse \\ \hline 
	status &  & final   \\ \hline 
	category & coding & system: \url{http://terminology.hl7.org/CodeSystem/observation-category} \\
\cline{3-3}
	& & code: vital-signs \\ \hline
	code & coding & system: Url von \ac{loinc}, \ac{snomedct}, und / oder \ac{ieee} \\ 
	\cline{3-3} 
	 &  & code: \ac{loinc}, \ac{snomedct}, oder \ac{ieee} \\ \hline
	\textbf{subject} & reference & Pseudonymisierte Patientennummer: \texttt{co6\_data\_string.val} wobei \texttt{co6\_data\_decimal\_6\_3.varID} = 1 und \texttt{co6\_data\_string.varID} = 8 \\ \hline
	 \textbf{valueQuantity}  & \textbf{value} & Wert der Messung: \texttt{
co6\_data\_decimal\_6\_3.val} \\
        \cline{2-3}
         & system & http://unitsofmeasure.org \\
         \cline{2-3}
         & \textbf{code} & cm - \texttt{co6\_config\_variables.unit} \\ \hline
     \textbf{effectiveDateTime}  & & Datum und Uhrzeit der Messung: \texttt{
co6\_data\_decimal\_6\_3.datetimeto} \\ \hline
\end{longtable}


\subsection{Körpertemperatur Blut} 
Profil: \url{https://www.medizininformatik-initiative.de/Kerndatensatz/Modul_Intensivmedizin/Koerpertemperatur-Blut--Observation.html}

\begin{longtable}{|l|l|p{7.5cm}|}
        \hline
        \rowcolor{lightgray} \multicolumn{3}{|l|}{Data Mapping (inhaltlich)} \\ \hline
        \textbf{id} &  & \texttt{co6\_data\_decimal\_6\_3.id} \\ \hline
	meta & profile & https://www.medizininformatik-initiative.de/fhir/ext/modul-icu/StructureDefinition/ koerpertemperatur-blut \\ \hline 
	status &  & final   \\ \hline 
	category & coding & system: \url{http://terminology.hl7.org/CodeSystem/observation-category} \\
\cline{3-3}
	& & code: vital-signs \\ \hline
	code & coding & system: Url von \ac{loinc}, \ac{snomedct}, und / oder \ac{ieee} \\ 
	\cline{3-3} 
	 &  & code: \ac{loinc}, \ac{snomedct}, oder \ac{ieee} \\ \hline
	 \textbf{subject} & reference & Pseudonymisierte Patientennummer: \texttt{co6\_data\_string.val} wobei \texttt{co6\_data\_decimal\_6\_3.varID} = 1 und \texttt{co6\_data\_string.varID} = 8 \\ \hline
	 \textbf{valueQuantity}  & \textbf{value} & Wert der Messung: \texttt{
co6\_data\_decimal\_6\_3.val} \\
        \cline{2-3}
         & system & http://unitsofmeasure.org \\
         \cline{2-3}
         & \textbf{code} & Cel - \texttt{co6\_config\_variables.unit} \\ \hline
     \textbf{effectiveDateTime}  & & Datum und Uhrzeit der Messung: \texttt{
co6\_data\_decimal\_6\_3.datetimeto} \\ \hline
\end{longtable}

\subsection{Körpertemperatur Kern} 
Profil: \url{https://www.medizininformatik-initiative.de/Kerndatensatz/Modul_Intensivmedizin/KoerpertemperaturKernObservation.html}

\begin{longtable}{|l|l|p{7.5cm}|}
	\hline
	\rowcolor{lightgray} \multicolumn{3}{|l|}{Data Mapping (inhaltlich)} \\ \hline
	\textbf{id} &  & \texttt{co6\_data\_decimal\_6\_3.id} \\ \hline
	meta & profile & https://www.medizininformatik-initiative.de/fhir/ext/modul-icu/StructureDefinition/ koerpertemperatur-kern \\ \hline 
	status &  & final   \\ \hline 
	category & coding & system: \url{http://terminology.hl7.org/CodeSystem/observation-category} \\
	\cline{3-3}
	& & code: vital-signs \\ \hline
	code & coding & system: Url von \ac{loinc}, \ac{snomedct}, und / oder \ac{ieee} \\ 
	\cline{3-3} 
	&  & code: \ac{loinc}, \ac{snomedct}, oder \ac{ieee} \\ \hline
	\textbf{subject} & reference & Pseudonymisierte Patientennummer: \texttt{co6\_data\_string.val} wobei \texttt{co6\_data\_decimal\_6\_3.varID} = 1 und \texttt{co6\_data\_string.varID} = 8 \\ \hline
	\textbf{valueQuantity}  & \textbf{value} & Wert der Messung: \texttt{
		co6\_data\_decimal\_6\_3.val} \\
	\cline{2-3}
	& system & http://unitsofmeasure.org \\
	\cline{2-3}
	& \textbf{code} & Cel - \texttt{co6\_config\_variables.unit} \\ \hline
	\textbf{effectiveDateTime}  & & Datum und Uhrzeit der Messung: \texttt{
		co6\_data\_decimal\_6\_3.datetimeto} \\ \hline
\end{longtable}

\subsection{Körpertemperatur nasal} 
Profil: \url{https://www.medizininformatik-initiative.de/Kerndatensatz/Modul_Intensivmedizin/Koerpertemperatur-nasal--Observation.html}

\begin{longtable}{|l|l|p{7.5cm}|}
	\hline
	\rowcolor{lightgray} \multicolumn{3}{|l|}{Data Mapping (inhaltlich)} \\ \hline
	\textbf{id} &  & \texttt{co6\_data\_decimal\_6\_3.id} \\ \hline
	meta & profile & https://www.medizininformatik-initiative.de/fhir/ext/modul-icu/StructureDefinition/ koerpertemperatur-nasal \\ \hline 
	status &  & final   \\ \hline 
	category & coding & system: \url{http://terminology.hl7.org/CodeSystem/observation-category} \\
	\cline{3-3}
	& & code: vital-signs \\ \hline
	code & coding & system: Url von \ac{loinc}, \ac{snomedct}, und / oder \ac{ieee} \\ 
	\cline{3-3} 
	&  & code: \ac{loinc}, \ac{snomedct}, oder \ac{ieee} \\ \hline
	\textbf{subject} & reference & Pseudonymisierte Patientennummer: \texttt{co6\_data\_string.val} wobei \texttt{co6\_data\_decimal\_6\_3.varID} = 1 und \texttt{co6\_data\_string.varID} = 8 \\ \hline
	\textbf{valueQuantity}  & \textbf{value} & Wert der Messung: \texttt{
		co6\_data\_decimal\_6\_3.val} \\
	\cline{2-3}
	& system & http://unitsofmeasure.org \\
	\cline{2-3}
	& \textbf{code} & Cel - \texttt{co6\_config\_variables.unit} \\ \hline
	\textbf{effectiveDateTime}  & & Datum und Uhrzeit der Messung: \texttt{
		co6\_data\_decimal\_6\_3.datetimeto} \\ \hline
\end{longtable}

\subsection{Körpertemperatur rektal} 
Profil: \url{https://www.medizininformatik-initiative.de/Kerndatensatz/Modul_Intensivmedizin/Koerpertemperatur-rektal--Observation.html}

\begin{longtable}{|l|l|p{7.5cm}|}
	\hline
	\rowcolor{lightgray} \multicolumn{3}{|l|}{Data Mapping (inhaltlich)} \\ \hline
	\textbf{id} &  & \texttt{co6\_data\_decimal\_6\_3.id} \\ \hline
	meta & profile & https://www.medizininformatik-initiative.de/fhir/ext/modul-icu/StructureDefinition/ koerpertemperatur-rektal \\ \hline 
	status &  & final   \\ \hline 
	category & coding & system: \url{http://terminology.hl7.org/CodeSystem/observation-category} \\
	\cline{3-3}
	& & code: vital-signs \\ \hline
	code & coding & system: Url von \ac{loinc}, \ac{snomedct}, und / oder \ac{ieee} \\ 
	\cline{3-3} 
	&  & code: \ac{loinc}, \ac{snomedct}, oder \ac{ieee} \\ \hline
	\textbf{subject} & reference & Pseudonymisierte Patientennummer: \texttt{co6\_data\_string.val} wobei \texttt{co6\_data\_decimal\_6\_3.varID} = 1 und \texttt{co6\_data\_string.varID} = 8 \\ \hline
	\textbf{valueQuantity}  & \textbf{value} & Wert der Messung: \texttt{
		co6\_data\_decimal\_6\_3.val} \\
	\cline{2-3}
	& system & http://unitsofmeasure.org \\
	\cline{2-3}
	& \textbf{code} & Cel - \texttt{co6\_config\_variables.unit} \\ \hline
	\textbf{effectiveDateTime}  & & Datum und Uhrzeit der Messung: \texttt{
		co6\_data\_decimal\_6\_3.datetimeto} \\ \hline
\end{longtable}


\subsection{Körpertemperatur Speiseröhre} 
Profil: \url{https://www.medizininformatik-initiative.de/Kerndatensatz/Modul_Intensivmedizin/Koerpertemperatur-Speiseroehre--Observation.html}

\begin{longtable}{|l|l|p{7.5cm}|}
	\hline
	\rowcolor{lightgray} \multicolumn{3}{|l|}{Data Mapping (inhaltlich)} \\ \hline
	\textbf{id} &  & \texttt{co6\_data\_decimal\_6\_3.id} \\ \hline
	meta & profile & https://www.medizininformatik-initiative.de/fhir/ext/modul-icu/StructureDefinition/
	koerpertemperatur-speiseroehre \\ \hline 
	status &  & final   \\ \hline 
	category & coding & system: \url{http://terminology.hl7.org/CodeSystem/observation-category} \\
	\cline{3-3}
	& & code: vital-signs \\ \hline
	code & coding & system: Url von \ac{loinc}, \ac{snomedct}, und / oder \ac{ieee} \\ 
	\cline{3-3} 
	&  & code: \ac{loinc}, \ac{snomedct}, oder \ac{ieee} \\ \hline
	\textbf{subject} & reference & Pseudonymisierte Patientennummer: \texttt{co6\_data\_string.val} wobei \texttt{co6\_data\_decimal\_6\_3.varID} = 1 und \texttt{co6\_data\_string.varID} = 8 \\ \hline
	\textbf{valueQuantity}  & \textbf{value} & Wert der Messung: \texttt{
		co6\_data\_decimal\_6\_3.val} \\
	\cline{2-3}
	& system & http://unitsofmeasure.org \\
	\cline{2-3}
	& \textbf{code} & Cel - \texttt{co6\_config\_variables.unit} \\ \hline
	\textbf{effectiveDateTime}  & & Datum und Uhrzeit der Messung: \texttt{
		co6\_data\_decimal\_6\_3.datetimeto} \\ \hline
\end{longtable}

\subsection{Körpertemperatur Trommelfell}

Profil: \url{https://www.medizininformatik-initiative.de/Kerndatensatz/Modul_Intensivmedizin/Koerpertemperatur-Trommelfell--Observation.html}

\begin{longtable}{|l|l|p{7.5cm}|}
	\hline
	\rowcolor{lightgray} \multicolumn{3}{|l|}{Data Mapping (inhaltlich)} \\ \hline
	\textbf{id} &  & \texttt{co6\_data\_decimal\_6\_3.id} \\ \hline
	meta & profile & https://www.medizininformatik-initiative.de/fhir/ext/modul-icu/StructureDefinition/ koerpertemperatur-trommelfel \\ \hline 
	status &  & final   \\ \hline 
	category & coding & system: \url{http://terminology.hl7.org/CodeSystem/observation-category} \\
	\cline{3-3}
	& & code: vital-signs \\ \hline
	code & coding & system: Url von \ac{loinc}, \ac{snomedct}, und / oder \ac{ieee} \\ 
	\cline{3-3} 
	&  & code: \ac{loinc}, \ac{snomedct}, oder \ac{ieee} \\ \hline
	\textbf{subject} & reference & Pseudonymisierte Patientennummer: \texttt{co6\_data\_string.val} wobei \texttt{co6\_data\_decimal\_6\_3.varID} = 1 und \texttt{co6\_data\_string.varID} = 8 \\ \hline
	\textbf{valueQuantity}  & \textbf{value} & Wert der Messung: \texttt{
		co6\_data\_decimal\_6\_3.val} \\
	\cline{2-3}
	& system & http://unitsofmeasure.org \\
	\cline{2-3}
	& \textbf{code} & Cel - \texttt{co6\_config\_variables.unit} \\ \hline
	\textbf{effectiveDateTime}  & & Datum und Uhrzeit der Messung: \texttt{
		co6\_data\_decimal\_6\_3.datetimeto} \\ \hline
\end{longtable}

\subsection{Maximaler Beatmungsdruck} 
 Profil: \url{https://www.medizininformatik-initiative.de/Kerndatensatz/Modul_Intensivmedizin/MaximalerBeatmungsdruckObservation.html}

\begin{longtable}{|l|l|p{7.5cm}|}
        \hline
        \rowcolor{lightgray} \multicolumn{3}{|l|}{Data Mapping (inhaltlich)} \\ \hline
        \textbf{id} &  & \texttt{co6\_data\_decimal\_6\_3.id} \\ \hline
	meta & profile & https://www.medizininformatik-initiative.de/fhir/ext/modul-icu/StructureDefinition/maximaler-beatmungsdruck \\ \hline 
	status &  & final   \\ \hline 
	category & coding & system: \url{http://snomed.info/sct} \\
\cline{3-3}
	& & code: 40617009 \\ \hline
	code & coding & system: Url von \ac{loinc}, \ac{snomedct}, und / oder \ac{ieee} \\ 
	\cline{3-3} 
	 &  & code: \ac{loinc}, \ac{snomedct}, oder \ac{ieee} \\ \hline
	 \textbf{subject} & reference & Pseudonymisierte Patientennummer: \texttt{co6\_data\_string.val} wobei \texttt{co6\_data\_decimal\_6\_3.varID} = 1 und \texttt{co6\_data\_string.varID} = 8 \\ \hline
	 \textbf{valueQuantity}  & \textbf{value} & Wert der Messung: \texttt{
co6\_data\_decimal\_6\_3.val} \\
        \cline{2-3}
         & system & http://unitsofmeasure.org \\
         \cline{2-3}
         & \textbf{code} & cm[H2O] - notwendige Transformation - \texttt{co6\_config\_variables.unit} = mmHg \\ \hline
     \textbf{effectiveDateTime}  & & Datum und Uhrzeit der Messung: \texttt{
co6\_data\_decimal\_6\_3.datetimeto} \\ \hline
\end{longtable}


\subsection{Mittlerer Beatmungsdruck} 

Profil: \url{https://www.medizininformatik-initiative.de/Kerndatensatz/Modul_Intensivmedizin/MittlererBeatmungsdruckObservation.html}

\begin{longtable}{|l|l|p{7.5cm}|}
	\hline
	\rowcolor{lightgray} \multicolumn{3}{|l|}{Data Mapping (inhaltlich)} \\ \hline
	\textbf{id} &  & \texttt{co6\_data\_decimal\_6\_3.id} \\ \hline
	meta & profile & https://www.medizininformatik-initiative.de/fhir/ext/modul-icu/StructureDefinition/mittlerer-beatmungsdruck \\ \hline 
	status &  & final   \\ \hline 
	category & coding & system: \url{http://snomed.info/sct} \\
	\cline{3-3}
	& & code: 40617009 \\ \hline
	code & coding & system: Url von \ac{loinc}, \ac{snomedct}, und / oder \ac{ieee} \\ 
	\cline{3-3} 
	&  & code: \ac{loinc}, \ac{snomedct}, oder \ac{ieee} \\ \hline
	\textbf{subject} & reference & Pseudonymisierte Patientennummer: \texttt{co6\_data\_string.val} wobei \texttt{co6\_data\_decimal\_6\_3.varID} = 1 und \texttt{co6\_data\_string.varID} = 8 \\ \hline
	\textbf{valueQuantity}  & \textbf{value} & Wert der Messung: \texttt{
		co6\_data\_decimal\_6\_3.val} \\
	\cline{2-3}
	& system & http://unitsofmeasure.org \\
	\cline{2-3}
	& \textbf{code} & cm[H2O] - notwendige Transformation - \texttt{co6\_config\_variables.unit} = mbar \\ \hline
	\textbf{effectiveDateTime}  & & Datum und Uhrzeit der Messung: \texttt{
		co6\_data\_decimal\_6\_3.datetimeto} \\ \hline
\end{longtable}

\subsection{Positiv-endexpiratorischer Druck} 

Profil: \url{https://www.medizininformatik-initiative.de/Kerndatensatz/Modul_Intensivmedizin/Positiv-endexpiratorischerDruckObservation.html}

\begin{longtable}{|l|l|p{7.5cm}|}
        \hline
        \rowcolor{lightgray} \multicolumn{3}{|l|}{Data Mapping (inhaltlich)} \\ \hline
        \textbf{id} &  & \texttt{co6\_data\_decimal\_6\_3.id} \\ \hline
	meta & profile & https://www.medizininformatik-initiative.de/fhir/ext/modul-icu/StructureDefinition/positiv-endexpiratorischer-druck \\ \hline 
	status &  & final   \\ \hline 
	category & coding & system: \url{http://snomed.info/sct} \\
\cline{3-3}
	& & code: 40617009 \\ \hline
	code & coding & system: Url von \ac{loinc}, \ac{snomedct}, und / oder \ac{ieee} \\ 
	\cline{3-3} 
	 &  & code: \ac{loinc}, \ac{snomedct}, oder \ac{ieee} \\ \hline
	 \textbf{subject} & reference & Pseudonymisierte Patientennummer: \texttt{co6\_data\_string.val} wobei \texttt{co6\_data\_decimal\_6\_3.varID} = 1 und \texttt{co6\_data\_string.varID} = 8 \\ \hline
	 \textbf{valueQuantity}  & value & Wert der Messung: \texttt{co6\_data\_decimal\_6\_3.val} \\
        \cline{2-3}
         & system & http://unitsofmeasure.org \\
         \cline{2-3}
         & code & cm[H2O] - notwendige Transformation - \texttt{co6\_config\_variables.unit} = mbar \\ \hline
     \textbf{effectiveDateTime}  & & Datum und Uhrzeit der Messung: \texttt{co6\_data\_decimal\_6\_3.datetimeto} \\ \hline
\end{longtable}


\subsection{Pulmonalarterieller wedge Blutdruck} 
Profil: \url{https://www.medizininformatik-initiative.de/Kerndatensatz/Modul_Intensivmedizin/PulmonalarteriellerwedgeBlutdruckObservation.html}

\begin{longtable}{|l|l|p{7.5cm}|}
        \hline
        \rowcolor{lightgray} \multicolumn{3}{|l|}{Data Mapping (inhaltlich)} \\ \hline
        \textbf{id} &  & \texttt{co6\_data\_decimal\_6\_3.id} \\ \hline
	meta & profile & https://www.medizininformatik-initiative.de/fhir/ext/modul-icu/StructureDefinition/pulmonalarterieller-wedge-druck \\ \hline 
	status &  & final   \\ \hline 
	category & coding & system: \url{http://terminology.hl7.org/CodeSystem/observation-category} \\
\cline{3-3}
	& & code: vital-signs\\ \hline
	code & coding & system: Url von \ac{loinc}, \ac{snomedct}, und / oder \ac{ieee} \\ 
	\cline{3-3} 
	 &  & code: \ac{loinc}, \ac{snomedct}, oder \ac{ieee} \\ \hline
	 \textbf{subject} & reference & Pseudonymisierte Patientennummer: \texttt{co6\_data\_string.val} wobei \texttt{co6\_data\_decimal\_6\_3.varID} = 1 und \texttt{co6\_data\_string.varID} = 8 \\ \hline
	 \textbf{valueQuantity}  & \textbf{value} & Wert der Messung: \texttt{
co6\_data\_decimal\_6\_3.val} \\
        \cline{2-3}
         & system & http://unitsofmeasure.org \\
         \cline{2-3}
         & \textbf{code} & mm[Hg] - \texttt{co6\_config\_variables.unit} \\ \hline
     \textbf{effectiveDateTime}  & & Datum und Uhrzeit der Messung: \texttt{
co6\_data\_decimal\_6\_3.datetimeto} \\
   \hline
\end{longtable}

\subsection{Pulmonalvaskulärer Widerstandsindex} 

Profil: \url{https://www.medizininformatik-initiative.de/Kerndatensatz/Modul_Intensivmedizin/PulmonalvaskulrerWiderstandsindexObservation.html}

\begin{longtable}{|l|l|p{7.5cm}|}
	\hline
	\rowcolor{lightgray} \multicolumn{3}{|l|}{Data Mapping (inhaltlich)} \\ \hline
	\textbf{id} &  & \texttt{co6\_data\_decimal\_6\_3.id} \\ \hline
	meta & profile & https://www.medizininformatik-initiative.de/fhir/ext/modul-icu/StructureDefinition/
	pulmonalvaskulaerer-widerstandsindex \\ \hline 
	status &  & final   \\ \hline 
	category & coding & system: \url{http://terminology.hl7.org/CodeSystem/observation-category} \\
	\cline{3-3}
	& & code: vital-signs\\ \hline
	code & coding & system: Url von \ac{loinc}, \ac{snomedct}, und / oder \ac{ieee} \\ 
	\cline{3-3} 
	&  & code: \ac{loinc}, \ac{snomedct}, oder \ac{ieee} \\ \hline
	\textbf{subject} & reference & Pseudonymisierte Patientennummer: \texttt{co6\_data\_string.val} wobei \texttt{co6\_data\_decimal\_6\_3.varID} = 1 und \texttt{co6\_data\_string.varID} = 8 \\ \hline
	\textbf{valueQuantity}  & \textbf{value} & Wert der Messung: \texttt{
		co6\_data\_decimal\_6\_3.val} \\
	\cline{2-3}
	& system & http://unitsofmeasure.org \\
	\cline{2-3}
	& \textbf{code} & mm[Hg] - \texttt{co6\_config\_variables.unit} \\ \hline
	\textbf{effectiveDateTime}  & & Datum und Uhrzeit der Messung: \texttt{
		co6\_data\_decimal\_6\_3.datetimeto} \\
	\hline
\end{longtable}


\subsection{Puls}

Profil: \url{https://www.medizininformatik-initiative.de/Kerndatensatz/Modul_Intensivmedizin/Puls--Observation.html}

\begin{longtable}{|l|l|p{7.5cm}|}
	\hline
	\rowcolor{lightgray} \multicolumn{3}{|l|}{Data Mapping (inhaltlich)} \\ \hline
	\textbf{id} &  & \texttt{co6\_data\_decimal\_6\_3.id} \\ \hline
	meta & profile & https://www.medizininformatik-initiative.de/fhir/ext/modul-icu/StructureDefinition/puls \\ \hline 
	status &  & final   \\ \hline 
	category & coding & system: \url{http://terminology.hl7.org/CodeSystem/observation-category} \\
	\cline{3-3}
	& & code: vital-signs\\ \hline
	code & coding & system: Url von \ac{loinc}, \ac{snomedct}, und / oder \ac{ieee} \\ 
	\cline{3-3} 
	&  & code: \ac{loinc}, \ac{snomedct}, oder \ac{ieee} \\ \hline
	\textbf{subject} & reference & Pseudonymisierte Patientennummer: \texttt{co6\_data\_string.val} wobei \texttt{co6\_data\_decimal\_6\_3.varID} = 1 und \texttt{co6\_data\_string.varID} = 8 \\ \hline
	\textbf{valueQuantity}  & \textbf{value} & Wert der Messung: \texttt{co6\_data\_decimal\_6\_3.val} \\
	\cline{2-3}
	& system & http://unitsofmeasure.org \\
	\cline{2-3}
	& \textbf{code} & /min - \texttt{co6\_config\_variables.unit} \\ \hline
	\textbf{effectiveDateTime}  & & Datum und Uhrzeit der Messung: \texttt{
		co6\_data\_decimal\_6\_3.datetimeto} \\
	\hline
\end{longtable}

\subsection{Sauerstoffgasfluss} 
Profil: \url{https://www.medizininformatik-initiative.de/Kerndatensatz/Modul_Intensivmedizin/SauerstoffgasflussObservation.html}

\begin{longtable}{|l|l|p{7.5cm}|}
        \hline
        \rowcolor{lightgray} \multicolumn{3}{|l|}{Data Mapping (inhaltlich)} \\ \hline
        \textbf{id} &  & \texttt{co6\_data\_decimal\_6\_3.id} \\ \hline
	meta & profile & https://www.medizininformatik-initiative.de/fhir/ext/modul-icu/StructureDefinition/sauerstoffgasfluss \\ \hline 
	status &  & final   \\ \hline 
	category & coding & system: \url{http://snomed.info/sct} \\
\cline{3-3}
	& & code: 182744004 \\ \hline
	code & coding & system: Url von \ac{loinc}, \ac{snomedct}, und / oder \ac{ieee} \\ 
	\cline{3-3} 
	 &  & code: \ac{loinc}, \ac{snomedct}, oder \ac{ieee} \\ \hline
	 \textbf{subject} & reference & Pseudonymisierte Patientennummer: \texttt{co6\_data\_string.val} wobei \texttt{co6\_data\_decimal\_6\_3.varID} = 1 und \texttt{co6\_data\_string.varID} = 8 \\ \hline
	 \textbf{valueQuantity}  & \textbf{value} & Wert der Messung: \texttt{
co6\_data\_decimal\_6\_3.val} \\
        \cline{2-3}
         & system & http://unitsofmeasure.org \\
         \cline{2-3}
         & \textbf{code} & L/min - \texttt{co6\_config\_variables.unit} \\ \hline
     \textbf{effectiveDateTime}  & & Datum und Uhrzeit der Messung: \texttt{
co6\_data\_decimal\_6\_3.datetimeto} \\
    \hline
\end{longtable}


\subsection{Sauerstoffsättigung im art. Blut durch Pulsoxymetrie} 
Profil: \url{https://www.medizininformatik-initiative.de/Kerndatensatz/Modul_Intensivmedizin/Sauerstoffsttigungimart.BlutdurchPulsoxymetrieObs.html}

\begin{longtable}{|l|l|p{7.5cm}|}
        \hline
        \rowcolor{lightgray} \multicolumn{3}{|l|}{Data Mapping (inhaltlich)} \\ \hline
        \textbf{id} &  & \texttt{co6\_data\_decimal\_6\_3.id} \\ \hline
	meta & profile & https://www.medizininformatik-initiative.de/fhir/ext/modul-icu/StructureDefinition/sauerstoffsaettigung-im-arteriellen-blut-durch-pulsoxymetrie \\ \hline 
	status &  & final   \\ \hline 
	category & coding & system: \url{http://terminology.hl7.org/CodeSystem/observation-category} \\
\cline{3-3}
	& & code: vital-signs \\ \hline
	code & coding & system: Url von \ac{loinc}, \ac{snomedct}, und / oder \ac{ieee} \\ 
	\cline{3-3} 
	 &  & code: \ac{loinc}, \ac{snomedct}, oder \ac{ieee} \\ \hline
	\textbf{subject} & reference & Pseudonymisierte Patientennummer: \texttt{co6\_data\_string.val} wobei \texttt{co6\_data\_decimal\_6\_3.varID} = 1 und \texttt{co6\_data\_string.varID} = 8 \\ \hline
	 \textbf{valueQuantity}  & \textbf{value} & Wert der Messung: \texttt{
co6\_data\_decimal\_6\_3.val} \\
        \cline{2-3}
         & system & http://unitsofmeasure.org \\
         \cline{2-3}
         & \textbf{code} & \% - \texttt{co6\_config\_variables.unit} \\ \hline
     \textbf{effectiveDateTime}  & & Datum und Uhrzeit der Messung: \texttt{
co6\_data\_decimal\_6\_3.datetimeto} \\
     \hline
\end{longtable}


\subsection{Spontane Atemfrequenz Beatmet} 

Profil: \url{https://www.medizininformatik-initiative.de/Kerndatensatz/Modul_Intensivmedizin/Spontane-Atemfrequenz-BeatmetObservation.html}

\begin{longtable}{|l|l|p{7.5cm}|}
        \hline
        \rowcolor{lightgray} \multicolumn{3}{|l|}{Data Mapping (inhaltlich)} \\ \hline
        \textbf{id} &  & \texttt{co6\_data\_decimal\_6\_3.id} \\ \hline
	meta & profile & https://www.medizininformatik-initiative.de/fhir/ext/modul-icu/StructureDefinition/spontane-atemfrequenz-beatmet \\ \hline 
	status &  & final   \\ \hline 
	category & coding & system: \url{http://snomed.info/sct} \\
\cline{3-3}
	& & code: 40617009 \\ \hline
	code & coding & system: Url von \ac{loinc}, \ac{snomedct}, und / oder \ac{ieee} \\ 
	\cline{3-3} 
	 &  & code: \ac{loinc}, \ac{snomedct}, oder \ac{ieee} \\ \hline
	 \textbf{subject} & reference & Pseudonymisierte Patientennummer: \texttt{co6\_data\_string.val} wobei \texttt{co6\_data\_decimal\_6\_3.varID} = 1 und \texttt{co6\_data\_string.varID} = 8 \\ \hline
	 \textbf{valueQuantity}  & \textbf{value} & Wert der Messung: \texttt{
co6\_data\_decimal\_6\_3.val} \\
        \cline{2-3}
         & system & http://unitsofmeasure.org \\
         \cline{2-3}
         & \textbf{code} & /min - notwendige Transformation - \texttt{co6\_config\_variables.unit} = bpm \\ \hline
     \textbf{effectiveDateTime}  & & Datum und Uhrzeit der Messung: \texttt{
co6\_data\_decimal\_6\_3.datetimeto} \\ \hline
\end{longtable}


\subsection{Spontane Mechanische Atemfrequenz Beatmet} 
Profil: \url{https://www.medizininformatik-initiative.de/Kerndatensatz/Modul_Intensivmedizin/Spontane-Mechanische-Atemfrequenz-BeatmetObservation.html}

\begin{longtable}{|l|l|p{7.5cm}|}
        \hline
        \rowcolor{lightgray} \multicolumn{3}{|l|}{Data Mapping (inhaltlich)} \\ \hline
        \textbf{id} &  & \texttt{co6\_data\_decimal\_6\_3.id} \\ \hline
	meta & profile & https://www.medizininformatik-initiative.de/fhir/ext/modul-icu/StructureDefinition/spontane-mechanische-atemfrequenz-beatmet \\ \hline 
	status &  & final   \\ \hline 
	category & coding & system: \url{http://snomed.info/sct} \\
\cline{3-3}
	& & code: 40617009 \\ \hline
	code & coding & system: Url von \ac{loinc}, \ac{snomedct}, und / oder \ac{ieee} \\ 
	\cline{3-3} 
	 &  & code: \ac{loinc}, \ac{snomedct}, oder \ac{ieee} \\ \hline
	 \textbf{subject} & reference & Pseudonymisierte Patientennummer: \texttt{co6\_data\_string.val} wobei \texttt{co6\_data\_decimal\_6\_3.varID} = 1 und \texttt{co6\_data\_string.varID} = 8 \\ \hline
	 \textbf{valueQuantity}  & \textbf{value} & Wert der Messung: \texttt{
co6\_data\_decimal\_6\_3.val} \\
        \cline{2-3}
         & system & http://unitsofmeasure.org \\
         \cline{2-3}
         & \textbf{code} & /min - notwendige Transformation - \texttt{co6\_config\_variables.unit} = bpm \\ \hline
     \textbf{effectiveDateTime}  & & Datum und Uhrzeit der Messung: \texttt{
co6\_data\_decimal\_6\_3.datetimeto} \\ \hline
\end{longtable}


\subsection{Substituatfluss} 
 Profil: \url{https://www.medizininformatik-initiative.de/Kerndatensatz/Modul_Intensivmedizin/SubstituatflussObservation.html}

\begin{longtable}{|l|l|p{7.5cm}|}
        \hline
        \rowcolor{lightgray} \multicolumn{3}{|l|}{Data Mapping (inhaltlich)} \\ \hline
        \textbf{id} &  & \texttt{co6\_data\_decimal\_6\_3.id} \\ \hline
	meta & profile & https://www.medizininformatik-initiative.de/fhir/ext/modul-icu/StructureDefinition/substituatfluss \\ \hline 
	status &  & final   \\ \hline 
	category & coding & system: \url{http://snomed.info/sct} \\
\cline{3-3}
	& & code: 182744004 \\ \hline
	code & coding & system: Url von \ac{loinc}, \ac{snomedct}, und / oder \ac{ieee} \\ 
	\cline{3-3} 
	 &  & code: \ac{loinc}, \ac{snomedct}, oder \ac{ieee} \\ \hline
	 \textbf{subject} & reference & Pseudonymisierte Patientennummer: \texttt{co6\_data\_string.val} wobei \texttt{co6\_data\_decimal\_6\_3.varID} = 1 und \texttt{co6\_data\_string.varID} = 8 \\ \hline
	 \textbf{valueQuantity}  & \textbf{value} & Wert der Messung: \texttt{
co6\_data\_decimal\_6\_3.val} \\
        \cline{2-3}
         & system & http://unitsofmeasure.org \\
         \cline{2-3}
         & \textbf{code} & mL/h - \texttt{co6\_config\_variables.unit} \\ \hline
     \textbf{effectiveDateTime}  & & Datum und Uhrzeit der Messung: \texttt{co6\_data\_decimal\_6\_3.datetimeto} \\ \hline
\end{longtable}

\subsection{Systemischer vaskulärer Widerstandsindex} 
Profil: \url{https://www.medizininformatik-initiative.de/Kerndatensatz/Modul_Intensivmedizin/SystemischervaskulrerWiderstandsindexObservation.html}

\begin{longtable}{|l|l|p{7.5cm}|}
	\hline
	\rowcolor{lightgray} \multicolumn{3}{|l|}{Data Mapping (inhaltlich)} \\ \hline
	\textbf{id} &  & \texttt{co6\_data\_decimal\_6\_3.id} \\ \hline
	meta & profile & https://www.medizininformatik-initiative.de/fhir/ext/modul-icu/StructureDefinition/systemischer-vaskulaerer-widerstandsindex \\ \hline 
	status &  & final   \\ \hline 
	category & coding & system: \url{http://terminology.hl7.org/CodeSystem/observation-category} \\
	\cline{3-3}
	& & code: vital-signs \\ \hline
	code & coding & system: Url von \ac{loinc}, \ac{snomedct}, und / oder \ac{ieee} \\ 
	\cline{3-3} 
	&  & code: \ac{loinc}, \ac{snomedct}, oder \ac{ieee} \\ \hline
	\textbf{subject} & reference & Pseudonymisierte Patientennummer: \texttt{co6\_data\_string.val} wobei \texttt{co6\_data\_decimal\_6\_3.varID} = 1 und \texttt{co6\_data\_string.varID} = 8 \\ \hline
	\textbf{valueQuantity}  & \textbf{value} & Wert der Messung: \texttt{
		co6\_data\_decimal\_6\_3.val} \\
	\cline{2-3}
	& system & http://unitsofmeasure.org \\
	\cline{2-3}
	& \textbf{code} & dyn.s/cm5/m2 - \texttt{co6\_config\_variables.unit} \\ \hline
	\textbf{effectiveDateTime}  & & Datum und Uhrzeit der Messung: \texttt{
		co6\_data\_decimal\_6\_3.datetimeto} \\
	\hline
\end{longtable}

\subsection{Unterstützungsdruck Beatmung} 
Profil: \url{https://www.medizininformatik-initiative.de/Kerndatensatz/Modul_Intensivmedizin/UntersttzungsdruckBeatmungObservation.html}

\begin{longtable}{|l|l|p{7.5cm}|}
	\hline
	\rowcolor{lightgray} \multicolumn{3}{|l|}{Data Mapping (inhaltlich)} \\ \hline
	\textbf{id} &  & \texttt{co6\_data\_decimal\_6\_3.id} \\ \hline
	meta & profile & https://www.medizininformatik-initiative.de/fhir/ext/modul-icu/StructureDefinition/
	unterstuetzungsdruck-beatmung \\ \hline 
	status &  & final   \\ \hline 
	category & coding & system: \url{http://snomed.info/sct} \\
	\cline{3-3}
	& & code: 40617009 \\ \hline
	code & coding & system: Url von \ac{loinc}, \ac{snomedct}, und / oder \ac{ieee} \\ 
	\cline{3-3} 
	&  & code: \ac{loinc}, \ac{snomedct}, oder \ac{ieee} \\ \hline
	\textbf{subject} & reference & Pseudonymisierte Patientennummer: \texttt{co6\_data\_string.val} wobei \texttt{co6\_data\_decimal\_6\_3.varID} = 1 und \texttt{co6\_data\_string.varID} = 8 \\ \hline
	\textbf{valueQuantity}  & \textbf{value} & Wert der Messung: \texttt{
		co6\_data\_decimal\_6\_3.val} \\
	\cline{2-3}
	& system & http://unitsofmeasure.org \\
	\cline{2-3}
	& \textbf{code} & cm[H2O] - notwendige Transformation - \texttt{co6\_config\_variables.unit} = mbar
	\\ \hline
	\textbf{effectiveDateTime}  & & Datum und Uhrzeit der Messung: \texttt{
		co6\_data\_decimal\_6\_3.datetimeto} \\
	\hline
\end{longtable}

\subsection{Venöser Druck} 

Profil: \url{https://www.medizininformatik-initiative.de/Kerndatensatz/Modul_Intensivmedizin/VenserDruckObservation.html}

\begin{longtable}{|l|l|p{7.5cm}|}
        \hline
        \rowcolor{lightgray} \multicolumn{3}{|l|}{Data Mapping (inhaltlich)} \\ \hline
        \textbf{id} &  & \texttt{co6\_data\_decimal\_6\_3.id} \\ \hline
	meta & profile & https://www.medizininformatik-initiative.de/fhir/ext/modul-icu/StructureDefinition/venoeser-druck \\ \hline 
	status &  & final   \\ \hline 
	category & coding & system: \url{http://snomed.info/sct} \\
\cline{3-3}
	& & code: 182744004 \\ \hline
	code & coding & system: Url von \ac{loinc}, \ac{snomedct}, und / oder \ac{ieee} \\ 
	\cline{3-3} 
	 &  & code: \ac{loinc}, \ac{snomedct}, oder \ac{ieee} \\ \hline
	 	\textbf{subject} & reference & Pseudonymisierte Patientennummer: \texttt{co6\_data\_string.val} wobei \texttt{co6\_data\_decimal\_6\_3.varID} = 1 und \texttt{co6\_data\_string.varID} = 8 \\ \hline
	 \textbf{valueQuantity}  & \textbf{value} & Wert der Messung: \texttt{
co6\_data\_decimal\_6\_3.val} \\
        \cline{2-3}
         & system & http://unitsofmeasure.org \\
         \cline{2-3}
         & \textbf{code} & mm[Hg] - \texttt{co6\_config\_variables.unit} \\ \hline
     \textbf{effectiveDateTime}  & & Datum und Uhrzeit der Messung: \texttt{co6\_data\_decimal\_6\_3.datetimeto} \\ \hline
\end{longtable}

\subsection{Zentralvenöser Druck} 

Profil: \url{https://www.medizininformatik-initiative.de/Kerndatensatz/Modul_Intensivmedizin/Zentralven-ser-Druck--Observation.html}

\begin{longtable}{|l|l|p{7.5cm}|}
        \hline
        \rowcolor{lightgray} \multicolumn{3}{|l|}{Data Mapping (inhaltlich)} \\ \hline
        \textbf{id} &  & \texttt{co6\_data\_decimal\_6\_3.id} \\ \hline
	meta & profile & https://www.medizininformatik-initiative.de/fhir/ext/modul-icu/StructureDefinition/zentralvenoeser-blutdruck \\ \hline 
	status &  & final   \\ \hline 
	category & coding & system: \url{http://terminology.hl7.org/CodeSystem/observation-category} \\
\cline{3-3}
	& & code: vital-signs \\ \hline
	code & coding & system: Url von \ac{loinc}, \ac{snomedct}, und / oder \ac{ieee} \\ 
	\cline{3-3} 
	 &  & code: \ac{loinc}, \ac{snomedct}, oder \ac{ieee} \\ \hline
	 \textbf{subject}  & reference & Pseudonymisierte Patientennummer: \texttt{co6\_medic\_patient.patid} \\ \hline
	 \textbf{valueQuantity}  & \textbf{value} & Wert der Messung: \texttt{co6\_data\_decimal\_6\_3.val} \\
        \cline{2-3}
         & system & http://unitsofmeasure.org \\
         \cline{2-3}
         & \textbf{code} & mm[Hg] - \texttt{co6\_config\_variables.unit} \\ \hline
     \textbf{effectiveDateTime}  & & Datum und Uhrzeit der Messung: \texttt{
co6\_data\_decimal\_6\_3.datetimeto} \\ \hline
\end{longtable}
